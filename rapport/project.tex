\documentclass[11pt,a4paper]{article}

\usepackage{amsmath}
\usepackage{amssymb}
\usepackage{amsthm}
\usepackage{mathtools}
\usepackage{enumerate}
\usepackage[T1]{fontenc}
\usepackage[hidelinks]{hyperref}
\usepackage{natbib}
\usepackage{xfrac}
\usepackage{caption}
\usepackage{threeparttable}
\usepackage{graphicx}
\usepackage{listings}
\usepackage{tikz}
\usetikzlibrary{calc,automata,positioning}
\usepackage{subcaption}

\usepackage[french]{babel}
\usepackage[utf8]{inputenc}

\newtheorem{prop}{Proposition}
\newtheorem{lemma}{Lemme}
\newtheorem{thm}{Théorème}
\newtheorem{definition}{Définition}
\newtheorem{conj}{Conjecture}
\newtheorem*{rem*}{Remarque}

\title{À propos du problème de finitude des groupes d'automates biréversibles}
\author{Maxime Flin \& Tristan François}

\begin{document}
\maketitle

\section*{Introduction}
Les automates de Mealy sont une extension des automates qui écrivent des mots dans un alphabet en même temps qu'ils en lisent dans ce même alphabet. À partir de l'action de ces automates sur l'ensemble des mots, on peut dégager une structure de (semi-)groupe engendré par l'automate. Pour étudier cette classe de (semi-)groupes, on peut travailler sur des constructions spécifiques à la structure sous-jacente d'automate (comme la minimisation ou la dualisation) pour résoudre des problèmes d'algèbre de la théorie des groupes.

Le problème de théorie des groupes auquel nous nous intéressons ici est le problème de finitude, c’est-à-dire savoir si le groupe engendré par un automate de Mealy donné est fini ou non. C'est un problème difficile, indécidable en général pour les semi-groupes d'automates~\cite{Undecidable}. Picantin et Klimann~\cite{DBLP:journals/corr/abs-1105-4725} ont proposé la $\mathfrak{md}$-réduction (définition \ref{def:md-red}), une procédure alternant minimisation et dualisation de l'automate et Klimann~\cite{Klimann13} a réussi à montrer que les groupes engendrés par des automates biréversibles (définition \ref{def:birev}) à deux états ou deux lettres sont finis si et seulement si l'automate est $\mathfrak{md}$-trivial.

Notre projet consistait à essayer d'étendre ce résultat à tous les automates biréversibles de plus de deux états/lettres. Cependant,  le critère de $\mathfrak{md}$-réduction n'est pas suffisant pour les automates à 4 états et 3 lettres (voir les quatre contres-exemples figure \ref{fig:fantastiques}). Picantin a toutefois remarqué que ces automates se factorisent et que l'automate obtenu en inversant les deux termes de la factorisation est quant à lui $\mathfrak{md}$-trivial. Nous nous sommes basés sur ces exemples pour ajouter l'opération de conjugaison (définition \ref{def:conj}) à la $\mathfrak{md}$-réduction afin de l'étendre en $\mathfrak{mdc}$-réduction (définition \ref{def:mdc-red}). Nous avons émis l'hypothèse que ce nouveau critère résoudrait le problème de finitude pour les classes d'automates biréversibles à plus de deux états/lettres.

Nous avons mené un travail théorique pour se convaincre du bien-fondé de notre hypothèse (sections \ref{sec:action} et \ref{sec:cloture}). Parallèlement à cela, nous avons été amenés à imaginer et implémenter des techniques efficaces, notamment pour engendrer des classes d'automates qui nous intéressent (section \ref{sec:gen}) et les factoriser (section \ref{sec:facto}) -- question sur laquelle aucun vrai résultat n'existe encore -- afin de tester empiriquement notre hypothèse. Grâce à cela, nous avons finalement pu réfuter notre hypothèse en exhibant un contre-exemple (figure \ref{fig:contre-exemple}).

\section{Automates de Mealy}

\subsection{Automates de Mealy et quelques constructions}

La notion d'automate de Mealy étend la simple notion d'automate en lui ajoutant une sortie en écriture.

\begin{definition}
  Un \textbf{automate de Mealy} est la donné d'un quadruplet $\left(\mathcal{Q}, \Sigma, \delta, \rho\right)$ où $\mathcal{Q}$ est l'ensemble des états de l'automate, $\Sigma$ l'alphabet sur lequel l'automate agit, $\delta$ une famille d'applications de $\mathcal{Q}$ dans $\mathcal{Q}$ indexées par $\Sigma$ qui représente les transitions entre les états et $\rho$ une famille d'applications de $\Sigma$ dans $\Sigma$ indexées par $\mathcal{Q}$ qui représente l'écriture en sortie de l'automate.
\end{definition}

La figure \ref{fig:example} montre une machine de Mealy qui incrémente un entier écrit en binaire. Une flèche du type $p\overset{x|y}{\longrightarrow}q$ indique que quand on est dans l'état $p$ et qu'on lit la lettre $x$, on se déplace dans l'état $q$ et on écrit la lettre $y$.

\begin{figure}[!ht]
  \begin{center}
    \begin{tikzpicture}[shorten >=1pt,node distance=2cm,on grid,auto]
      \node[state] (q_0) {$0$};
      \node[state] (q_1) [right=of q_0] {$1$};
      \path[->,>=latex]
      (q_0) edge [loop above] node {$1|0$} (q_0)
      edge node {$0|1$} (q_1)
      (q_1) edge [loop above] node {$0|0~1|1$} (q_1);
    \end{tikzpicture}
    \caption{une machine de Mealy $\mathcal{A}$ \label{fig:example}}
  \end{center}
\end{figure}

On peut représenter l'action d'un automate de Mealy par son diagramme en croix:

\begin{figure}[!ht]
  \begin{center}
    \begin{tikzpicture}
      \draw[->,>=latex] (-1.25, 0.5) node[above] {1} to (-1.25, -0.5) node[below] {0};
      \draw[->,>=latex] (-1.5, 0) node[left] {0} to (-1, 0) node[right] {0};

      \draw[->,>=latex] (-0.25, 0.5) node[above] {0} to (-0.25, -0.5) node[below] {1};
      \draw[->,>=latex] (-0.5, 0) to (0, 0) node[right] {1};

      \draw[->,>=latex] (0.75, 0.5) node[above] {1} to (0.75, -0.5) node[below] {1};
      \draw[->,>=latex] (0.5, 0) to (1, 0) node[right] {1};
    \end{tikzpicture}
    \caption{action de l'état 0 de la machine $\mathcal{A}$ sur le mot 101. Sont représentés horizontalement les états sucessifs et verticalement l'action de chacun de ces états sur les lettres du mot.}
  \end{center}
\end{figure}

Une autre représentation qui sera commode dans la partie \ref{sec:gen} est de le représenter par un graphe en hélice. On représente l'automate par un graphe dont les sommets sont étiquetés par des couples états lettres. Pour chaque flèche du type $p\overset{x|y}{\longrightarrow}q$ dans l'automate, il y a une flèche de l'état $p, x$ vers l'état $q, y$ dans le graphe en hélice. Aussi remarque-t-on que de chaque sommet du graphe en hélice part exactement une flèche.

\begin{figure}[!ht]
  \begin{center}
    \begin{tikzpicture}[shorten >=1pt,node distance=2cm,on grid,auto]
      \node[state] (q_00) {$0, 0$};
      \node[state] (q_11) [right=of q_00] {$1, 1$};
      \node[state] (q_10) [above=of q_11] {$1, 0$};
      \node[state] (q_01) [above=of q_00] {$0, 1$};
      \path[->,>=latex]
      (q_00) edge node {} (q_11)
      (q_10) edge node {} (q_01)
      (q_11) edge [loop right] node {} (q_11)
      (q_01) edge node {} (q_00);
    \end{tikzpicture}
    \caption{graphe en hélice de $\mathcal{A}$}
  \end{center}
\end{figure}

\begin{definition}
  Soit $\mathcal{A}=\left(\mathcal{Q}, \Sigma, \delta, \rho\right)$ un automate de Mealy.
  \begin{enumerate}[(i)]
  \item On dit que $\mathcal{A}$ est \textbf{inversible} si $\rho_q$ est une permutation de $\Sigma$ pour tout état $q\in\mathcal{Q}$.
  \item On dit que $\mathcal{A}$ est \textbf{réversible} si $\delta_x$ est une permuration de $\mathcal{Q}$ pour toute lettre $x\in\Sigma$.
  \end{enumerate}
\end{definition}

\begin{definition}
  Si un machine de Mealy $\mathcal{M}$ est inversible, alors on peut construire sont \textbf{\textit{inverse}}, noté $\mathcal{M}^{-1}$, en remplacant les flèches $p\overset{x|y}{\longrightarrow}q$ par $p^{-1}\overset{y|x}{\longrightarrow}q^{-1}$.
\end{definition}


\begin{figure}[h!]
  \centering
  \scalebox{.9} {
  \begin{subfigure}[b]{0.5\textwidth}
    \centering
    \begin{tikzpicture}[shorten >=1pt,node distance=2cm,on grid,auto]
      \node[state] (p) {$p$};
      \node[state] (q) [right=of q_0] {$q$};
      \path[->,>=latex]
      (p) edge [loop left] node {$1|0$} (p)
      (p) edge [bend left] node {$0|1$} (q)
      (q) edge [bend left] node {$0|0$} (p)
      (q) edge [loop right] node {$1|1$} (q);
    \end{tikzpicture}
  \end{subfigure}
  ~
  \begin{subfigure}[b]{0.5\textwidth}
    \centering
    \begin{tikzpicture}[shorten >=1pt,node distance=2cm,on grid,auto]
      \node[state] (p) {$p^{-1}$};
      \node[state] (q) [right=of q_0] {$q^{-1}$};
      \path[->,>=latex]
      (p) edge [loop left] node {$0|1$} (p)
      (p) edge [bend left] node {$1|0$} (q)
      (q) edge [bend left] node {$0|0$} (p)
      (q) edge [loop right] node {$1|1$} (q);
    \end{tikzpicture}
  \end{subfigure}
  }
  \caption{Automate et son inverse}
\end{figure}


\begin{definition}
  Si une machine de Mealy est réversible, alors on peut construire sont \textbf{\textit{dual}}, noté $\mathfrak{d}\mathcal{M}$, en remplacant les flèches $p\overset{x|y}{\longrightarrow}q$ par $x\overset{p|q}{\longrightarrow}y$.
\end{definition}


\begin{figure}[h!]
  \centering
  \scalebox{.9}{
  \begin{subfigure}[b]{0.5\textwidth}
    \centering
    \begin{tikzpicture}[shorten >=1pt,node distance=2cm,on grid,auto]
      \node[state] (p) {$p$};
      \node[state] (q) [right=of p] {$q$};
      \path[->,>=latex]
      (p) edge [loop left] node {$1|0$} (p)
      (p) edge [bend left] node {$0|1$} (q)
      (q) edge [bend left] node {$0|0$} (p)
      (q) edge [loop right] node {$1|1$} (q);
    \end{tikzpicture}
  \end{subfigure}
  ~
  \begin{subfigure}[b]{0.5\textwidth}
    \centering
    \begin{tikzpicture}[shorten >=1pt,node distance=2cm,on grid,auto]
      \node[state] (p) {$0$};
      \node[state] (q) [right=of p] {$1$};
      \path[->,>=latex]
      (p) edge [loop left] node {$q|p$} (p)
      (p) edge [bend left] node {$p|q$} (q)
      (q) edge [bend left] node {$q|q$} (p)
      (q) edge [loop right] node {$p|p$} (q);
    \end{tikzpicture}
  \end{subfigure}
  }
  \caption{Automate et son dual}
\end{figure}


\begin{definition} \label{def:birev}
  On dit que $\mathcal{A}$ est \textbf{biréversible} s'il est inversible, réversible et que son inverse est réversible.
\end{definition}

\begin{definition}
  Soit $\mathcal{A}=\left(\mathcal{Q}, \Sigma, \delta, \rho\right)$ un automate de Mealy. \textbf{L'équivalence de Nérode} $\equiv$ sur $\mathcal{Q}$ est la limite de la suite de relations d'équivalence $\equiv_k$ de plus en plus fines définies récursivement par

  \begin{align*}
    \forall p, q \in Q,\\
    p \equiv_0 q &\iff \rho_p = \rho_q \\
    \forall k \geq 0, p \equiv_{k+1}q &\iff \left(p\equiv_kq \wedge \forall x \in \Sigma,~\delta_x(p)=\delta_x(q)\right)
  \end{align*}

  Le \textbf{minimisé} de $\mathcal{A}$ est l'automate $\mathfrak{m}(\mathcal{A})=\left(\sfrac{\mathcal{Q}}{\equiv}, \Sigma, \bar{\delta}, \bar{\rho}\right)$, où pour tout $p\in\mathcal{Q}$ et $x\in\Sigma$, $\bar{\delta}_x([p]) = [\delta_x(p)]$ et $\bar{\rho}_{[p]} =  \rho_p$.
\end{definition}

\begin{definition}
  \label{def:produit}
  Soient $\mathcal{A} = \left(\mathcal{Q}, \Sigma, \delta, \rho\right)$ et $\mathcal{B} = \left(\mathcal{Q'}, \Sigma, \delta', \rho'\right)$ deux automates sur le même alphabet. On appelle \textbf{\textit{automate produit}} la machine $\mathcal{AB} = \left(\mathcal{Q}\times\mathcal{Q'}, \Sigma, \delta'', \rho''\right)$ avec

\[ \delta_x''(p_0, p_1) = (\delta_x(p), \delta_{\rho_{p_0}(x)}'(q))\]
et
\[ \rho_{(p_0,p_1)}(x) = \rho_{p_1}'(\rho_{p_0}(x)). \]
\end{definition}

\begin{figure}[h!]
  \begin{subfigure}[b]{0.5\textwidth}
    \centering
    \begin{tikzpicture}
      \draw [->,>=latex] (-0.5, 1) node[left] {$p_0$} to (0.5, 1) node[right] {$q_0$};
      \draw [->,>=latex] (0, 1.5) node[above] {$x$} to (0, 0.5);

      \draw [->,>=latex] (-0.5, -0.5) node[left] {$p_1$} to (0.5, -0.5) node[right] {$q_1$};
      \draw [->,>=latex] (0, 0) node[above] {$z$} to (0, -1) node[below] {$y$};
    \end{tikzpicture}
    \caption{En haut l'automate $\mathcal{A}$ et en bas $\mathcal{B}$}
  \end{subfigure}
  ~
  \begin{subfigure}[b]{0.5\textwidth}
    \centering
    \begin{tikzpicture}
      \draw [->,>=latex] (-0.5, 0) node[left] {$p_0p_1$} to (0.5, 0) node[right] {$q_0q_1$};
      \draw [->,>=latex] (0, 0.5) node[above] {$x$} to (0, -0.5) node[below] {$y$};
    \end{tikzpicture}
    \caption{Une flèche de l'automate $\mathcal{AB}$}
  \end{subfigure}
  \caption{Correspondance entre les flèches dans $\mathcal{AB}$ et celles de $\mathcal{A}$ et de $\mathcal{B}$}
\end{figure}

\begin{definition}{\cite{DBLP:journals/corr/abs-1105-4725}}
  \label{def:md-red}
  Une paire d'automates duaux est \textbf{$\mathfrak{md}$-réduite} si les deux automates sont minimaux. La \textbf{$\mathfrak{md}$-réduction} d'un automate consiste à minimiser cet automate et son dual jusqu'à avoir une paire $\mathfrak{md}$-réduite.

  On dit qu'un automate est \textbf{$\mathfrak{md}$-trivial} si son $\mathfrak{md}$-réduit est l'automate trivial.
\end{definition}

\begin{figure}[!ht]
  \centering
  \scalebox{0.8}{
    \begin{tikzpicture}
      \coordinate (A2) at (-2,-7);
      \coordinate (B2) at (-2,-11);
      \coordinate (C2) at (2,-7);
      \coordinate (D2) at (2,-11);
      \node[draw,thick,circle,minimum width = 2.5em] (QA2) at (A2) {$b$};
      \node[draw,thick,circle,minimum width = 2.5em] (QB2) at (B2) {$a$};
      \node[draw,thick,circle,minimum width = 2.5em] (QC2) at (C2) {$c$};
      \node[draw,thick,circle,minimum width = 2.5em] (QD2) at (D2) {$d$};
      \draw[thick,->,>=latex] (QA2) to[out=115,in=155,looseness=6] node[midway, above left]{$1|2$} (QA2);
      \draw[thick,->,>=latex] (QA2) to[bend left] node[midway, above]{\begin{tabular}{c} $0|1$ \\ $2|0$ \end{tabular}} (QC2);
      \draw[thick,->,>=latex] (QB2) to[out=205,in=245,looseness=6] node[midway, below left]{$1|2$} (QB2);
      \draw[thick,->,>=latex] (QB2) to[bend left] node[midway, above]{\begin{tabular}{c} $0|0$ \\ $2|1$ \end{tabular}} (QD2);
      \draw[thick,->,>=latex] (QC2) to[out=25,in=65,looseness=6] node[midway, above right]{$1|2$} (QC2);
      \draw[thick,->,>=latex] (QC2) to[bend left] node[midway, below]{\begin{tabular}{c} $0|1$ \\ $2|0$ \end{tabular}} (QA2);
      \draw[thick,->,>=latex] (QD2) to[out=295,in=335,looseness=6] node[midway, below right]{$1|2$} (QD2);
      \draw[thick,->,>=latex] (QD2) to[bend left] node[midway, below]{\begin{tabular}{c} $0|0$ \\ $2|1$ \end{tabular}} (QB2);

      % minization
      \draw[thick,->,>=latex] (0, -13) to node[midway, right] {$\mathfrak{m}$} (0, -14);

      % minimisé du conjugué

      \coordinate (A2) at (2,-17);
      \coordinate (B2) at (-2,-17);
      \node[draw,thick,circle,minimum width = 2.5em] (QA2) at (A2) {$bc$};
      \node[draw,thick,circle,minimum width = 2.5em] (QB2) at (B2) {$ad$};
      \draw[thick,->,>=latex] (QA2) to[out=70,in=110,looseness=6] node[midway, above]{\begin{tabular}{c} $0|0$ \\ $1|2$ \\ $2|1$ \end{tabular}} (QA2);
      \draw[thick,->,>=latex] (QB2) to[out=70,in=110,looseness=6] node[midway, above]{\begin{tabular}{c} $0|1$ \\ $1|2$ \\ $2|0$ \end{tabular}} (QB2);

      % dualisation

      \draw[thick,<->,>=latex] (3.5, -16) to node[midway, above] {$\mathfrak{d}$} (4.5, -16);



    \coordinate (A2) at (6.5,-17);
    \coordinate (B2) at (9.5,-17);
    \coordinate (C2) at ($(A2) + (60:3)$);
    \node[draw,thick,circle,minimum width = 2.5em] (QA2) at (A2) {$0$};
    \node[draw,thick,circle,minimum width = 2.5em] (QB2) at (B2) {$2$};
    \node[draw,thick,circle,minimum width = 2.5em] (QC2) at (C2) {$1$};

    \draw[thick,->,>=latex] (QA2) to[out=190,in=230,looseness=6] node[midway, below left]{$ad|ad$} (QA2);
    \draw[thick,->,>=latex] (QA2) to node[midway, above left] {$bc|bc$} (QC2);
    \draw[thick,->,>=latex] (QB2) to node[midway, below] {$bc|bc$} (QA2);
    \draw[thick,->,>=latex] (QB2) to[bend left] node[midway, below left] {$ad|ad$} (QC2);
    \draw[thick,->,>=latex] (QC2) to[bend left] node[midway, above right] {\begin{tabular}{c} $ad|ad$ \\ $bc|bc$ \end{tabular}} (QB2);

    %minimisation
    \draw[thick,->,>=latex] (8, -19) to node[midway, right] {$\mathfrak{m}$} (8, -20);

    \coordinate (A2) at (8, -23);
    \node[draw,thick,circle,minimum width = 2.5em] (QA2) at (A2) {$012$};
    \draw[thick,->,>=latex] (QA2) to[out=70,in=110,looseness=6] node[midway, above]{\begin{tabular}{c} $ad|ad$ \\ $bc|bc$ \end{tabular}} (QA2);

    %dualisation

    \draw[thick,<->,>=latex] (3.5, -22) to node[midway, above] {$\mathfrak{d}$} (4.5, -22);

    \coordinate (A2) at (-1.5, -22);
    \coordinate (B2) at (1.5, -22);
    \node[draw,thick,circle,minimum width = 2.5em] (QA2) at (A2) {$ad$};
    \node[draw,thick,circle,minimum width = 2.5em] (QB2) at (B2) {$bc$};
    \draw[thick,->,>=latex] (QA2) to[out=70,in=110,looseness=6] node[midway, above]{$012|012$} (QA2);
    \draw[thick,->,>=latex] (QB2) to[out=70,in=110,looseness=6] node[midway, above]{$012|012$} (QB2);

    %minimisation
    \draw[thick,->,>=latex] (0, -23) to node[midway, right] {$\mathfrak{m}$} (0, -24);

    \coordinate (A2) at (0, -26);
    \node[draw,thick,circle,minimum width = 2.5em] (QA2) at (A2) {$adbc$};
    \draw[thick,->,>=latex] (QA2) to[out=70,in=110,looseness=6] node[midway, above]{$012|012$} (QA2);

  \end{tikzpicture}
  }
  \caption{$\mathfrak{md}$-réduction de l'automate $\mathcal{B}$.\label{fig:md-red-ex}}
\end{figure}


\subsection{Action sur les mots et (semi-)groupes engendrés\label{sec:action}}

\subsubsection*{Notations}
Soit $\Sigma$ un alphabet. On note $\Sigma^*$ l'ensemble des mots sur cet alphabet et $\varepsilon$ le mot vide.

\begin{definition}
  Soit $\mathcal{A}=\left(\mathcal{Q}, \Sigma, \delta, \rho\right)$ un automate de Mealy. On peut étendre l'ensemble de définition de $\rho$ à $\Sigma^*$ par induction comme suit.

  Soient $p\in\mathcal{Q}$, $x\in\Sigma$ et $\textbf{u}\in\Sigma^*$
  \begin{align*}
    &\rho_p(\epsilon)=\varepsilon \\
    &\rho_p(x\textbf{u})=\rho_p(x)\rho_{\delta_x(p)}(\textbf{u})
  \end{align*}

  On définit le \textbf{semi-groupe d'automate engendré par $\mathcal{A}$} comme
  \begin{equation*}
    \left<\mathcal{A}\right>_+=\left<\rho_p, \forall p\in\mathcal{Q}\right>
  \end{equation*}

  Si l'automate $\mathcal{A}$ est inversible, alors on peut aussi considérer l'inverse des $\rho_p$. Dans ce cas, l'automate engendre un groupe noté $\left<\mathcal{A}\right>$.

\end{definition}

\begin{prop}{\cite{DBLP:journals/corr/abs-1105-4725}}
  \label{prop:finitude-d}
  Soit $\mathcal{A}$ un automate de Mealy. Le (semi-)groupe $\left<\mathcal{A}\right>_+$ est fini si et seulement si $\left<\mathfrak{d}\mathcal{A}\right>_+$ est fini.
\end{prop}

\begin{prop}
  \label{prop:finitude-m}
  Soit $\mathcal{A}$ un automate de Mealy, alors
  \[ \left<\mathfrak{m}\mathcal{A}\right> = \left<\mathcal{A}\right>. \]
\end{prop}

Des propositions \ref{prop:finitude-d} et \ref{prop:finitude-m} on déduit les propositions suivantes.

\begin{prop}
  \label{prop:md-trivial}
  Un automate de Mealy engendre un groupe fini si et seulement si son $\mathfrak{md}$-réduit engendre un groupe fini.
\end{prop}

\begin{prop}
  Tout automate $\mathfrak{md}$-trivial engendre un groupe fini.
\end{prop}

La réciproque est fausse en général (voir figure \ref{fig:fantastiques}). \footnote{``un mot sur la faible efficacité en dehors des biréversibles''}

Il semble pourtant que dans la classe particulière des automates biréversibles, la $\mathfrak{md}$-réduction est efficace.

\begin{thm}{\cite{Klimann13}}
  \label{thm:K}
  Tout automate à deux lettres et/ou deux états engendre un groupe fini si et seulement s'il est $\mathfrak{md}$-trivial.
\end{thm}

Ce théorème n'est pas valide pour les automates biréversibles en général.
\newpage

\begin{figure}[!ht]
  \begin{center}
    \scalebox{.6}{
      \begin{tikzpicture}[scale=0.9]
        \coordinate (A) at (-10,-2);
        \coordinate (B) at (-6,-2);
        \coordinate (C) at (-6,2);
        \coordinate (D) at (-10,2);
        \node[draw,circle,minimum width = 2.5em] (QA) at (A) {};
        \node[draw,circle,minimum width = 2.5em] (QB) at (B) {};
        \node[draw,circle,minimum width = 2.5em] (QC) at (C) {};
        \node[draw,circle,minimum width = 2.5em] (QD) at (D) {};
        \draw[->,>=latex] (QA) to[out=240,in=210,looseness=8] node[midway, below left]{$1|2$} (QA);
        \draw[->,>=latex] (QA) to[bend right] node[midway, below]{$0|1$} (QB);
        \draw[->,>=latex] (QA) to[bend right] node[midway, right]{$2|0$} (QD);
        \draw[->,>=latex] (QB) to[bend right] node[midway, above]{$0|0$} (QA);
        \draw[->,>=latex] (QB) to[out=330, in=300, looseness=8] node[midway, below right]{$1|2$} (QB);
        \draw[->,>=latex] (QB) to[bend right] node[midway, right]{$2|1$} (QC);
        \draw[->,>=latex] (QC) to[bend right] node[midway, left]{$2|0$} (QB);
        \draw[->,>=latex] (QC) to[out=30, in=60, looseness=8] node[midway, above right]{$0|2$} (QC);
        \draw[->,>=latex] (QC) to[bend right] node[midway, above]{$1|1$} (QD);
        \draw[->,>=latex] (QD) to[bend right] node[midway, left]{$2|1$} (QA);
        \draw[->,>=latex] (QD) to[bend right] node[midway, below]{$1|0$} (QC);
        \draw[->,>=latex] (QD) to[out=120, in=150, looseness=8] node[midway, above left]{$0|2$} (QD);


        \coordinate (A) at (0,0);
        \coordinate (B) at (0:3);
        \coordinate (C) at (120:3);
        \coordinate (D) at (240:3);
        \node[draw,circle,minimum width = 2.5em] (QA) at (A) {};
        \node[draw,circle,minimum width = 2.5em] (QB) at (B) {};
        \node[draw,circle,minimum width = 2.5em] (QC) at (C) {};
        \node[draw,circle,minimum width = 2.5em] (QD) at (D) {};
        \draw[->,>=latex] (QA) to[bend right] node[midway, below]{$1|0$} (QB);
        \draw[->,>=latex] (QB) to[bend right] node[midway, above]{$2|0$} (QA);
        \draw[->,>=latex] (QA) to[bend right] node[midway, above right]{$0|1$} (QC);
        \draw[->,>=latex] (QC) to[bend right] node[midway, below left]{$0|2$} (QA);
        \draw[->,>=latex] (QA) to[bend right] node[midway, above left]{$2|2$} (QD);
        \draw[->,>=latex] (QD) to[bend right] node[midway, below right]{$1|1$} (QA);
        \draw[->,>=latex] (QB) to[bend right] node[midway, below]{$1|2$} (QC);
        \draw[->,>=latex] (QC) to[bend left=40] node[midway, above]{$2|1$} (QB);
        \draw[->,>=latex] (QC) to[bend right] node[midway, right]{$1|0$} (QD);
        \draw[->,>=latex] (QD) to[bend left=40] node[midway, left]{$2|0$} (QC);
        \draw[->,>=latex] (QD) to[bend right] node[midway, above]{$0|2$} (QB);
        \draw[->,>=latex] (QB) to[bend left=40] node[midway, below]{$0|1$} (QD);

        \coordinate (A) at (6,-2);
        \coordinate (B) at (10,-2);
        \coordinate (C) at (10,2);
        \coordinate (D) at (6,2);
        \node[draw,circle,minimum width = 2.5em] (QA) at (A) {};
        \node[draw,circle,minimum width = 2.5em] (QB) at (B) {};
        \node[draw,circle,minimum width = 2.5em] (QC) at (C) {};
        \node[draw,circle,minimum width = 2.5em] (QD) at (D) {};
        \draw[->,>=latex] (QA) to[out=240,in=210,looseness=8] node[midway, below left]{$1|0$} (QA);
        \draw[->,>=latex] (QA) to[bend right] node[midway, below]{$0|2$} (QB);
        \draw[->,>=latex] (QA) to[bend right] node[midway, right]{$2|1$} (QD);
        \draw[->,>=latex] (QB) to[bend right] node[midway, above]{$0|2$} (QA);
        \draw[->,>=latex] (QB) to[out=330, in=300, looseness=8] node[midway, below right]{$1|1$} (QB);
        \draw[->,>=latex] (QB) to[bend right] node[midway, right]{$2|0$} (QC);
        \draw[->,>=latex] (QC) to[bend right] node[midway, left]{$2|0$} (QB);
        \draw[->,>=latex] (QC) to[out=30, in=60, looseness=8] node[midway, above right]{$0|1$} (QC);
        \draw[->,>=latex] (QC) to[bend right] node[midway, above]{$1|2$} (QD);
        \draw[->,>=latex] (QD) to[bend right] node[midway, left]{$2|1$} (QA);
        \draw[->,>=latex] (QD) to[bend right] node[midway, below]{$1|2$} (QC);
        \draw[->,>=latex] (QD) to[out=120, in=150, looseness=8] node[midway, above left]{$0|0$} (QD);
      \end{tikzpicture}
    }
  \end{center}
  \caption{Les plus petits contre-exemples à une généralisation du théorème \ref{thm:K}. Le quatrième contre-exemple est l'automate inverse du premier.\label{fig:fantastiques}
  }
\end{figure}

On observe cependant que les contre-exemples de la figure \ref{fig:fantastiques} sont factorisables et qu'en permutant leurs facteurs on trouve un automate $\mathfrak{md}$-trivial (figure \ref{fig:mdc-red-ex}). Ce qui conduit à introduire les opérations suivantes.

\begin{definition}
  \label{def:conj}
  Deux automates $\mathcal{A}$ et $\mathcal{B}$ sont \textbf{\textit{conjugés}} s'il existe des automates $\mathcal{C}$ et $\mathcal{D}$ tels que $\mathcal{A}=\mathcal{CD}$ et $\mathcal{B}=\mathcal{DC}$.
\end{definition}

\begin{definition}
  \label{def:mdc-red}
  Une paire d'automates duaux est $\mathfrak{mdc}$-réduite si elle est $\mathfrak{md}$-réduite et que les deux automates sont irréductibles.
  La $\mathfrak{mdc}$-réduction consiste à répéter la $\mathfrak{md}$-réduction de l'automate et de ses conjugués jusqu'à trouver des paires $\mathfrak{mdc}$-réduites.
\end{definition}

\begin{rem*}
  Contrairement au $\mathfrak{md}$-réduit, il n'existe pas une unique $\mathfrak{mdc}$-réduction.
\end{rem*}

\begin{figure}[!ht]
\begin{center}
  \scalebox{0.7}{
  \begin{tikzpicture}

    % Automate de départ : fantastique2

    \coordinate (A1) at (0,0);
    \coordinate (B1) at (0:3);
    \coordinate (C1) at (120:3);
    \coordinate (D1) at (240:3);
    \node[draw,thick,circle,minimum width = 2.5em] (QA1) at (A1) {};
    \node[draw,thick,circle,minimum width = 2.5em] (QB1) at (B1) {};
    \node[draw,thick,circle,minimum width = 2.5em] (QC1) at (C1) {};
    \node[draw,thick,circle,minimum width = 2.5em] (QD1) at (D1) {};
    \draw[thick,->,>=latex] (QA1) to[bend right] node[midway, below]{$1|0$} (QB1);
    \draw[thick,->,>=latex] (QB1) to[bend right] node[midway, above]{$2|0$} (QA1);
    \draw[thick,->,>=latex] (QA1) to[bend right] node[midway, above right]{$0|1$} (QC1);
    \draw[thick,->,>=latex] (QC1) to[bend right] node[midway, below left]{$0|2$} (QA1);
    \draw[thick,->,>=latex] (QA1) to[bend right] node[midway, above left]{$2|2$} (QD1);
    \draw[thick,->,>=latex] (QD1) to[bend right] node[midway, below right]{$1|1$} (QA1);
    \draw[thick,->,>=latex] (QB1) to[bend right] node[midway, below]{$1|2$} (QC1);
    \draw[thick,->,>=latex] (QC1) to[bend left=40] node[midway, above]{$2|1$} (QB1);
    \draw[thick,->,>=latex] (QC1) to[bend right] node[midway, right]{$1|0$} (QD1);
    \draw[thick,->,>=latex] (QD1) to[bend left=40] node[midway, left]{$2|0$} (QC1);
    \draw[thick,->,>=latex] (QD1) to[bend right] node[midway, above]{$0|2$} (QB1);
    \draw[thick,->,>=latex] (QB1) to[bend left=40] node[midway, below]{$0|1$} (QD1);

    % égal

    \draw (4,0) node {$=$};

    % Décomposition en produit

    \coordinate (A2) at (6,2);
    \coordinate (B2) at (6,-2);
    \node[draw,thick,circle,minimum width = 2.5em] (QA2) at (A2) {};
    \node[draw,thick,circle,minimum width = 2.5em] (QB2) at (B2) {};
    \draw[thick,->,>=latex] (QA2) to[out=70,in=110,looseness=6] node[midway, above]{$0|2$} (QA2);
    \draw[thick,->,>=latex] (QA2) to[bend right] node[midway, left]{\begin{tabular}{c} $1|1$ \\ $2|0$ \end{tabular}} (QB2);
    \draw[thick,->,>=latex] (QB2) to[out=250,in=290,looseness=6] node[midway, below]{$0|2$} (QB2);
    \draw[thick,->,>=latex] (QB2) to[bend right] node[midway, right]{\begin{tabular}{c} $1|0$ \\ $2|1$ \end{tabular}} (QA2);

    \draw (8,0) node {$\times$};

    \coordinate (A3) at (10,2);
    \coordinate (B4) at (10,-2);
    \node[draw,thick,circle,minimum width = 2.5em] (QA3) at (A3) {};
    \node[draw,thick,circle,minimum width = 2.5em] (QB4) at (B4) {};
    \draw[thick,->,>=latex] (QA3) to[out=70,in=110,looseness=6] node[midway, above]{$1|0$} (QA3);
    \draw[thick,->,>=latex] (QA3) to[bend right] node[midway, left]{\begin{tabular}{c} $0|2$ \\ $2|1$ \end{tabular}} (QB4);
    \draw[thick,->,>=latex] (QB4) to[out=250,in=290,looseness=6] node[midway, below]{$1|0$} (QB4);
    \draw[thick,->,>=latex] (QB4) to[bend right] node[midway, right]{\begin{tabular}{c} $0|1$ \\ $2|2$ \end{tabular}} (QA3);

    % conjugaison

    \draw[thick,->,>=latex] (8, -4) to node[midway, left] {$\mathfrak{c}$} (8, -5);

    % produit conjugué

    \coordinate (A3) at (6,-7);
    \coordinate (B4) at (6,-11);
    \node[draw,thick,circle,minimum width = 2.5em] (QA3) at (A3) {};
    \node[draw,thick,circle,minimum width = 2.5em] (QB4) at (B4) {};
    \draw[thick,->,>=latex] (QA3) to[out=70,in=110,looseness=6] node[midway, above]{$1|0$} (QA3);
    \draw[thick,->,>=latex] (QA3) to[bend right] node[midway, left]{\begin{tabular}{c} $0|2$ \\ $2|1$ \end{tabular}} (QB4);
    \draw[thick,->,>=latex] (QB4) to[out=250,in=290,looseness=6] node[midway, below]{$1|0$} (QB4);
    \draw[thick,->,>=latex] (QB4) to[bend right] node[midway, right]{\begin{tabular}{c} $0|1$ \\ $2|2$ \end{tabular}} (QA3);

    \draw (8,-9) node {$\times$};

    \coordinate (A2) at (10,-7);
    \coordinate (B2) at (10,-11);
    \node[draw,thick,circle,minimum width = 2.5em] (QA2) at (A2) {};
    \node[draw,thick,circle,minimum width = 2.5em] (QB2) at (B2) {};
    \draw[thick,->,>=latex] (QA2) to[out=70,in=110,looseness=6] node[midway, above]{$0|2$} (QA2);
    \draw[thick,->,>=latex] (QA2) to[bend right] node[midway, left]{\begin{tabular}{c} $1|1$ \\ $2|0$ \end{tabular}} (QB2);
    \draw[thick,->,>=latex] (QB2) to[out=250,in=290,looseness=6] node[midway, below]{$0|2$} (QB2);
    \draw[thick,->,>=latex] (QB2) to[bend right] node[midway, right]{\begin{tabular}{c} $1|0$ \\ $2|1$ \end{tabular}} (QA2);

    %égalité

    \draw (4,-9) node {$=$};

    %conjugué

    \coordinate (A2) at (-2,-7);
    \coordinate (B2) at (-2,-11);
    \coordinate (C2) at (2,-7);
    \coordinate (D2) at (2,-11);
    \node[draw,thick,circle,minimum width = 2.5em] (QA2) at (A2) {$b$};
    \node[draw,thick,circle,minimum width = 2.5em] (QB2) at (B2) {$a$};
    \node[draw,thick,circle,minimum width = 2.5em] (QC2) at (C2) {$c$};
    \node[draw,thick,circle,minimum width = 2.5em] (QD2) at (D2) {$d$};
    \draw[thick,->,>=latex] (QA2) to[out=115,in=155,looseness=6] node[midway, above left]{$1|2$} (QA2);
    \draw[thick,->,>=latex] (QA2) to[bend left] node[midway, above]{\begin{tabular}{c} $0|1$ \\ $2|0$ \end{tabular}} (QC2);
    \draw[thick,->,>=latex] (QB2) to[out=205,in=245,looseness=6] node[midway, below left]{$1|2$} (QB2);
    \draw[thick,->,>=latex] (QB2) to[bend left] node[midway, above]{\begin{tabular}{c} $0|0$ \\ $2|1$ \end{tabular}} (QD2);
    \draw[thick,->,>=latex] (QC2) to[out=25,in=65,looseness=6] node[midway, above right]{$1|2$} (QC2);
    \draw[thick,->,>=latex] (QC2) to[bend left] node[midway, below]{\begin{tabular}{c} $0|1$ \\ $2|0$ \end{tabular}} (QA2);
    \draw[thick,->,>=latex] (QD2) to[out=295,in=335,looseness=6] node[midway, below right]{$1|2$} (QD2);
    \draw[thick,->,>=latex] (QD2) to[bend left] node[midway, below]{\begin{tabular}{c} $0|0$ \\ $2|1$ \end{tabular}} (QB2);
  \end{tikzpicture}
  }
\end{center}
  \caption{$\mathfrak{mdc}$-réduction du deuxième contre-exemple. Il est conjugé avec l'automate $\mathcal{B}$ de la figure \ref{fig:md-red-ex} qui est $\mathfrak{md}$-trvial.\label{fig:mdc-red-ex}}
\end{figure}
\newpage

\begin{conj}
  \label{conj:birev-mdc}
  Un automate de Mealy biréversible engendre un groupe fini si et seulement si son $\mathfrak{mdc}$-réduit est trivial.
\end{conj}

Cette conjecture est l'objet des recherches ménées au cours de ce projet.

\begin{prop}
  \label{prop:finitude-c}
  Soient $\mathcal{A}, \mathcal{B}$ des automates de Mealy.
  L'automate $\mathcal{A}\mathcal{B}$ engendre un groupe fini si et seulement si $\mathcal{B}\mathcal{A}$ engendre un groupe fini.
\end{prop}

\begin{proof}
  Soient $\mathcal{A}=\left(\mathcal{Q}_1, \Sigma, \delta, \rho\right)$ et $\mathcal{B}=\left(\mathcal{Q}_2, \Sigma, \delta, \rho\right)$ des automates de Mealy sur un même alphabet.

  On suppose sans perte de généralité $\mathcal{Q}_1$ et $\mathcal{Q}_2$ disjoints. On écrit donc $\rho$ pour les deux automates sans ambiguïté.

  On suppose $\left<\mathcal{A}\mathcal{B}\right>$ fini.

  Tout élément de $\left<\mathcal{B}\mathcal{A}\right>$ est de la forme
  \[
    \rho_{p_1q_1}\circ\rho_{p_2q_2}\circ\cdots\circ\rho_{p_nq_n}
  \]
  où les $p_i$ sont des éléments de $\mathcal{Q}_2$ et les $q_i$ des éléments de $\mathcal{Q}_1$. Or
  \begin{align*}
    \rho_{p_1q_1}\circ\rho_{p_2q_2}\circ\cdots\circ\rho_{p_nq_n} &= \rho_{q_1}\circ\rho_{p_1}\circ\rho_{q_2}\circ\rho_{p_2}\circ\cdots\circ\rho_{q_n}\circ\rho_{p_n} \\
    &=\rho_{q_1}\circ\left(\rho_{p_1}\circ\rho_{q_2}\circ\rho_{p_2}\circ\cdots\circ\rho_{q_n}\right)\circ\rho_{p_n} \\
    &=\rho_{q_1}\circ\underbrace{\left(\rho_{q_2p_1}\circ\rho_{q_3p_2}\circ\cdots\circ\rho_{q_np_{n-1}}\right)}_{\in\left<\mathcal{A}\mathcal{B}\right>}\circ\rho_{p_n}
  \end{align*}

  On en déduit que tout élément de $\left<\mathcal{B}\mathcal{A}\right>$ s'écrit comme la composition d'un $\rho_{q},~q\in\mathcal{Q}_1$, d'un $\rho_\bullet\in\left<\mathcal{A}\mathcal{B}\right>$ et d'un $\rho_{p},~p\in\mathcal{Q}_2$. Or $\mathcal{Q}_1$ et $\mathcal{Q}_2$ sont finis, donc le nombre de choix de $p$ et de $q$ est fini. De plus, on a supposé le nombre d'éléments $\rho_\bullet\in\left<\mathcal{A}\mathcal{B}\right>$ fini, donc le nombre d'élements qui s'écrivent \[ \rho_q\circ\rho_\bullet\circ\rho_p \] est fini. On a montré plus haut que tout élément de $\left<\mathcal{B}\mathcal{A}\right>$ sont de cette forme, on en conclut qu'il y en a un nombre fini.

  Les facteurs $\mathcal{A}$ et $\mathcal{B}$ jouant des rôles complètement symétriques, la même démonstration marche pour montrer la réciproque.
\end{proof}

On remarque toutefois que cette proposition n'est pas vraie pour les produits $\mathcal{ABC}$ et $\mathcal{ACB}$.

\begin{figure}[h!]
  CONTRE EXEMPLE
  \caption{Contre-exemple pour 3 facteurs}
\end{figure}

On déduit des propositions \ref{prop:finitude-d}, \ref{prop:finitude-m} et \ref{prop:finitude-c} la proposition suivante.

\begin{prop}
  Toute machine de Mealy engendre un (semi)groupe fini si et seulement si son $\mathfrak{mdc}$-réduit engendre un (semi)groupe fini.
\end{prop}

\begin{prop}
  Toute machine de Mealy $\mathfrak{mdc}$-trivial engendre un (semi)groupe fini.
\end{prop}

Nous avons donc prouvé une des implications de la conjecture~\ref{conj:birev-mdc}.

\section{Génération et factorisation d'automates de Mealy}
La génération de biréversibles et la factorisation d'automates de Mealy sont des problèmes pour lesquels aucune solution efficace n'existe pour l'instant. Dans le but de pouvoir infirmer ou consolider la conjecture \ref{conj:birev-mdc}, une partie de notre travail a été d'essayer de trouver et d'implémenter des méthodes de génération et de factorisations efficaces pour les automates de Mealy biréversibles.


\subsection{Clôture de classes d'automates\label{sec:cloture}}

Puisque nous nous intéressons à la classe des automates biréversibles, montrons qu'elle est bien close par les opérations qui nous intéressent, \emph{i.e.} la dualisation, la minimisation, le produit et la factorisation.

\begin{prop}[Clôture des inversibles par produit]
  Soient $\mathcal{A}$ et $\mathcal{B}$ des automates de Mealy \textbf{inversibles}. Alors $\mathcal{A}\cdot\mathcal{B}$ est inversible
\end{prop}

\begin{proof}
  Soient $\mathcal{A}=\left(\mathcal{Q}, \Sigma, \delta, \rho\right)$ et $\mathcal{B}=\left(\mathcal{Q'}, \Sigma, \delta', \rho'\right)$ deux automates inversibles, et soit $\mathcal{A\cdot B}=\left(\mathcal{Q\times Q'}, \Sigma, \delta'', \rho''\right)$ leur produit.


  Considérons un état $(p, r)$ de ce produit. Alors $\rho_r\circ\rho'_p=\rho_{(p,r)}$, or puisque les automates $\mathcal{A}$ et $\mathcal{B}$ sont inversibles, $\rho_p$ et $\rho'_r$ sont des permutations du groupe de symétrie de $\Sigma$, on en déduit que $\rho_{(p, r)}$ en est une aussi.

  On a bien que tous les ${(\rho''_q)}_{q\in Q\times Q'}$ sont des permutations sur les lettres, c'est à dire que l'automate est inversible.
\end{proof}

\begin{prop}[Clôture des réversibles par produit]
  Soient $\mathcal{A}$ et $\mathcal{B}$ des automates de Mealy \textbf{réversibles}. Si $\mathcal{A}$ est \textbf{inversible}, alors $\mathcal{A}\cdot\mathcal{B}$ est réversible.
\end{prop}

\begin{proof}
  Soient $\mathcal{A}=\left(\mathcal{Q}, \Sigma, \delta, \rho\right)$ et $\mathcal{B}=\left(\mathcal{Q'}, \Sigma, \delta', \rho'\right)$ deux automates réversibles, et soit $\mathcal{A\cdot B}=\left(\mathcal{Q\times Q'}, \Sigma, \delta'', \rho''\right)$ leur produit.


  On suppose $\mathcal{A}$ inversible.


  Par définition $\delta''_x(pr)=(\delta_x(p), \delta'_{\rho_p(x)}(r))$. Or, puisque les automates sont réversibles, les $(\delta_x)_{x\in\Sigma}$ et ${(\delta'_x)}_{x\in\Sigma}$ sont des bijections. De plus, $\mathcal{A}$ est inversible donc les ${(\rho_p)}_{p\in \mathcal{Q}}$ sont des bijections. On en conclut que les ${(\delta''_{pr})}_{pr\in\mathcal{Q}\times\mathcal{Q'}}$ sont aussi des bijections.
\end{proof}

\begin{rem*}
    Les biréversibles ne sont pas clôs par produit.
\end{rem*}

\begin{prop}[Clôture des inversibles par facteur]\label{prop_cloture_inv_facteurs}
  Soit $\mathcal{A}$ un automate de Mealy \textbf{inversible} qui se factorise en deux automates $\mathcal{A}_1$ et $\mathcal{A}_2$. Alors ces deux automates sont aussi inversibles.
\end{prop}

Avant de faire la démonstration, rappelons un résultat élémentaire qui nous sera utile.
\begin{lemma}\label{lem:bij}
  Soit $f:A\rightarrow B$ et $g:B\rightarrow C$ deux fonctions telles que $g\circ f$ soit une bijection.
  Alors
  \begin{enumerate}[(i)]
  \item $g$ est surjective
  \item $f$ est injective.
  \end{enumerate}
\end{lemma}

\begin{proof}
  \begin{enumerate}[(i)]
  \item Soit $x\in C$, alors il existe un $y\in A$ tel que
    \[ (g\circ f)(y) = x \]
    c'est à dire tel que
    \[ g(f(y)) = x. \]
    On en déduit immédiatement que $g$ est surjective.

  \item Soit $x,y \in B$ tels que \[ f(x) = f(y). \] Alors \[ g(f(x)) = g(f(y)) \] et on en déduit que $x = y$, c'est à dire que $f$ est injective.
  \end{enumerate}
\end{proof}

\begin{proof}[Démonstration de la proposition \ref{prop_cloture_inv_facteurs}]
  Soient $\mathcal{A}=\left(\mathcal{Q\cdot Q'}, \Sigma, \delta, \rho\right)$, $\mathcal{A}_1=\left(\mathcal{Q}, \Sigma, \delta', \rho'\right)$ et $\mathcal{A}_2=\left(\mathcal{Q'}, \Sigma, \delta'', \rho''\right)$ tel que ci dessus.

  Pour chaque état $(p, r)$ de $\mathcal{A}$, $\rho_{(p, r)}\in \mathfrak{S}(\Sigma)$. Comme $\mathcal{A}=\mathcal{A}_1\cdot\mathcal{A}_2$, on a l'égalité $\rho_{(p, r)}=\rho'_r\circ\rho''_p$.

  Or $\rho_{(p, r)}$ est une bijection donc $\rho'_r$ est sujective et $\rho''_p$ est injective (lemme \ref{lem:bij}). Or puisque ce sont des fonctions de $\Sigma$ dans $\Sigma$ qui est fini on a l'équivalence
  \[ injective \iff surjective \iff bijective. \]
  D'où le résultat.
\end{proof}

\begin{prop}\label{prop_inverse_produit}
    Soit $\mathcal{A}$ un automate de Mealy \textbf{inversible} qui se factorise en deux automates $\mathcal{A}_1$ et $\mathcal{A}_2$.
    Alors $\mathcal{A}^{-1} = \mathcal{A}_2^{-1} \cdot \mathcal{A}_1^{-1}$.
\end{prop}

\begin{proof}
  Notons $\rho$, $\rho_1$ et $\rho_2$ les fonctions associées aux machines $\mathcal{A}$, $\mathcal{A}_1$ et $\mathcal{A}_2$. Alors par définition \[ \rho = \rho_2\circ\rho_1 \] et donc que \[ \rho^{-1} = \left(\rho_2\circ\rho_1\right)^{-1} = \rho_1^{-1}\circ\rho_2^{-1}. \]

  $\rho_1^{-1}$ et $\rho_2^{-1}$ sont bien définies puisqu'on a montré précédemment (prop \ref{prop_cloture_inv_facteurs}) que les automates inversibles étaient clos par facteurs.
\end{proof}

\begin{prop}[Clôture des réversibles par facteur]\label{prop_cloture_rev_facteurs}
  Soit $\mathcal{A}$ un automate de Mealy \textbf{réversible} qui se factorise en deux automates $\mathcal{A}_1$ et $\mathcal{A}_2$. Alors
  \begin{itemize}
  \item $\mathcal{A}_1$ est réversible.
  \item si $\mathcal{A}_1$ est inversible, alors $\mathcal{A}_2$ est réversible.
  \end{itemize}
\end{prop}

\begin{proof}
  Soient $\mathcal{A}=\left(\mathcal{Q\cdot Q'}, \Sigma, \delta, \rho\right)$ réversible, $\mathcal{A}_1=\left(\mathcal{Q}, \Sigma, \delta', \rho'\right)$ et $\mathcal{A}_2=\left(\mathcal{Q'}, \Sigma, \delta'', \rho''\right)$ tels que $\mathcal{A} = \mathcal{A}_1\cdot\mathcal{A}_2$.

  Par définition, on a que $\delta_x(pr) = (\delta'_x(p), \delta''_{\rho'_p(x)}(r))$. Les ${(\delta_x)}_{x\in\Sigma}$ sont inversibles, alors il est clair les ${(\delta'_x)}_{x\in\Sigma}$ sont inversibles.

  De plus, chacun des $\delta''_{\rho_p(x)}$ est inversibles, donc si $\mathcal{A}_1$ est inversible, les ${(\rho_p)}_{p\in\Sigma}$ étant des bijections, alors tous les ${(\delta''_x)}_{x\in\Sigma}$ sont inversibles. D'où $\mathcal{A}_2$ est réversible.
\end{proof}

\begin{prop}[Clôture des biréversibles par facteur]
  Soit $\mathcal{A}$ un automate \textbf{biréversible} et $\mathcal{A}_1$, $\mathcal{A}_2$ des automates tels que $\mathcal{A}=\mathcal{A}_1\cdot\mathcal{A}_2$, alors $\mathcal{A}_1$ et $\mathcal{A}_2$ sont biréversibles.
\end{prop}

\begin{proof}
    $\mathcal{A}$ est biréversible, donc en particulier $\mathcal{A}$ est inversible et réversible. D'après la proposition \ref{prop_cloture_inv_facteurs}, $\mathcal{A}_1$ et $\mathcal{A}_2$ sont inversibles. Puisque $\mathcal{A}_1$ est inversible, d'après la proposition \ref{prop_cloture_rev_facteurs}, $\mathcal{A}_1$ et $\mathcal{A}_2$ sont réversibles.

    D'après la proposition \ref{prop_inverse_produit}, $\mathcal{A}^{-1} = \mathcal{A}_2^{-1} \cdot \mathcal{A}_1^{-1}$. Or $\mathcal{A}^{-1}$ est réversible puisque $\mathcal{A}$ est biréversible. Donc, toujours d'après la proposition \ref{prop_cloture_rev_facteurs} et puisque $\mathcal{A}_1^{-1}$ est inversible, $\mathcal{A}_1^{-1}$ et $\mathcal{A}_2^{-1}$ sont réversibles.

    Ainsi, $\mathcal{A}_1$ et $\mathcal{A}_2$ sont bien biréversibles.
\end{proof}

\subsection{Algorithme de génération des biréversibles\label{sec:gen}}

\subsubsection*{Représentation des machines de Mealy en mémoire}
Les machines de Mealy sont représentées en mémoire par deux matrices \lstinline$delta$ et \lstinline{rho}, l'une représentant les $(\delta_x)_{x\in\Sigma}$ et l'autre les $(\rho_p)_{p\in\mathcal{Q}}$. Dans les deux cas, les lignes des matrices sont indexées par les états et les colonnes par les lettres.

\begin{enumerate}[(i)]
\item On reconnaît une machine réversible en vérifiant que les colonnes de \textrm{delta}  sont des permutations.
\item On reconnaît une machine inversible en vérifiant que les lignes dans \textrm{rho} sont des permutations.
\end{enumerate}

\subsubsection*{L'algorithme}
Notre algorithme de génération se base sur le résultat suivant

\begin{prop}{\cite{DBLP:journals/corr/abs-1105-4725}}
  \label{thm:ir-helix}
  Soit $\mathcal{A}$ un automate de Mealy inversible réversible. Les propositions suivantes sont équivalentes~:

  \begin{enumerate}[(i)]
  \item $\mathcal{A}$ est biréversible
  \item Le graphe en hélice de $\mathcal{A}$ est une union de cycles disjoints.
  \end{enumerate}
\end{prop}

Notre algorithme procède donc à la recherche exhaustive de toutes les machines de Mealy inversibles réversibles en maintenant la contrainte que le graphe en hélice est une union de cycles disjoints.

Nous maintenons une liste de \textit{sources} et de \textit{cibles} qui représentent les sommets du graphe en hélice desquels encore aucune flèche ne part et les sommets du graphe en hélice vers lesquels encore aucune flèche ne pointe. En choisissant nos sommets dans ces ensembles nous nous assurons que le graphe en hélice obtenu est bien une union de cycles disjoints. À chaque fois que nous modifions \textrm{delta} ou \textrm{rho}, nous vérifions que la machine est bien inversible-réversible.

Ainsi, les automates trouvés -- en vertu de la proposition \ref{thm:ir-helix} -- sont bien biréversibles. La structure globale du programme est représentée sur la figure~\ref{fig:gen-pseudo-code}.

\begin{figure}[!ht]
\begin{verbatim}
def rec(start, prev, sources, targets, delta, rho):
    if not sources and not targets:
        # plus de sources ni de targets
        # on a fini la génération et on retourne l'automate
        return delta, rho

    # s'il n'y a pas de précédent on commence un nouveau cycle
    # alors on choisit un sommets dans les sources
    # et on commence un nouveau cycle
    if not prev:
        start = sources.pop()

    [...]

    # on backtrack sur les targets
    for _ in range(len(targets)):
        # on choisit le sommet suivant dans le cycle
        # puis on modifie l'automate en conséquence
        p_next, x_next = targets.pop(0)
        delta[p_prev][x_prev] = p_next
        rho[p_prev][x_prev] = x_next

        [...]
        # on s'assure que l'automate est encore
        # inversible-réversible et si c'est le cas
        # on lance la récursion pour explorer cette direction
        if valid_delta(delta) and valid_rho(rho):
          res = rec(...)

    [...]

\end{verbatim}
  \caption{Structure globale de la fonction de génération en python\label{fig:gen-pseudo-code}}
\end{figure}

\subsubsection*{À isomorphisme près}

On a rapidement constaté que le nombre d'automates à générer est très important. Aussi, il ne semble pas réaliste de tester la $\mathfrak{mdc}$-réduction sur tous ceux-ci. On a donc cherché à extraire seulement les classes d'isomorphismes au sens de Nérode de notre génération.

La bibliothèque de graphe \textrm{Nauty}~\cite{Nauty} est une des plus efficaces. Il a fallu s'assurer que les morphismes de \textit{graphe} trouvé par \textrm{Nauty} correspondent bien à des morphismes de \textit{machine de Mealy}. Nous avons donc utilisé ce que nous appelons \textit{\textbf{le graphe en hélice augmenté}}: le graphe en hélice auquel nous ajoutons des sommets pour chaque état et chaque lettre de l'automate et les relions à ceux qu'ils étiquettent. Ainsi, si un morphisme de graphe trouvé par Nauty permute deux lettres ou deux états, il les permute sur tous les sommets du graphe en hélice.

\begin{figure}[!ht]
  \begin{center}
    \begin{tikzpicture}[scale=0.7]
        \coordinate (A) at (-10, 0);
        \coordinate (B) at (-7, 0);

        \node[draw,circle,minimum width = 2.5em] (q_0) at (A) {$a$};
        \node[draw,circle,minimum width = 2.5em] (q_1) at (B) {$b$};
        \path[->,>=latex]
        (q_0) edge [loop left] node {$0|1$} (q_0)
        (q_0) edge [bend left] node[midway, above] {$1|1$} (q_1)
        (q_1) edge [bend left] node[midway, below] {$0|0$} (q_0)
        (q_1) edge [loop right] node {$1|0$} (q_1);

        %############################################

        \coordinate (A) at (-1,-1);
        \coordinate (B) at (1,-1);
        \coordinate (C) at (1,1);
        \coordinate (D) at (-1,1);

        \coordinate (P1) at (-3,0);
        \coordinate (P2) at (3,0);
        \coordinate (PA) at (0,-3);
        \coordinate (PB) at (0,3);

        \node[draw,circle,minimum width = 2.5em] (QA) at (A) {$0|a$};
        \node[draw,circle,minimum width = 2.5em] (QB) at (B) {$1|a$};
        \node[draw,circle,minimum width = 2.5em] (QC) at (C) {$1|b$};
        \node[draw,circle,minimum width = 2.5em] (QD) at (D) {$0|b$};

        \node[draw,circle,minimum width = 2.5em,fill=black!20!green] (F1) at (P1) {$0$};
        \node[draw,circle,minimum width = 2.5em,fill=black!20!green] (F2) at (P2) {$1$};
        \node[draw,circle,minimum width = 2.5em,fill=white!20!orange] (FA) at (PA) {$a$};
        \node[draw,circle,minimum width = 2.5em,fill=white!20!orange] (FB) at (PB) {$b$};

        \draw[->,>=latex] (QA) to[bend right] (QB);
        \draw[->,>=latex] (QB) to[bend right] (QC);
        \draw[->,>=latex] (QC) to[bend right] (QD);
        \draw[->,>=latex] (QD) to[bend right] (QA);

        \draw[->,>=latex] (F1) to[bend right] (QA);
        \draw[->,>=latex] (F1) to[bend left] (QD);
        \draw[->,>=latex] (F2) to[bend left] (QB);
        \draw[->,>=latex] (F2) to[bend right] (QC);

      \draw[->,>=latex] (FA) to[bend right] (QB);
      \draw[->,>=latex] (FA) to[bend left] (QA);
      \draw[->,>=latex] (FB) to[bend left] (QC);
      \draw[->,>=latex] (FB) to[bend right] (QD);
    \end{tikzpicture}
  \end{center}
  \caption{Une machine de Mealy et son graphe en hélice augmenté\label{fig:helix-aug}}
\end{figure}

Pour calculer les isomorphismes entre les graphes, \textrm{Nauty} associe à un graphe \textit{\textbf{une forme canonique}}, \emph{i.e.} un graphe représentant de sa classe d'isomorphisme. Nous prenons donc comme représentant des classes d'isomorphismes de machines de Mealy celles qui sont une forme canonique au sens de \textrm{Nauty}. Lors de la génération, nous nous contentons de calculer la forme canonique du graphe en hélice augmenté puis de tester s'il laisse les lettres et les états inchangés. Auquel cas, il est canonique.

Nauty permet de préciser des classes d'équivalence sur les sommets du graphe qui doivent être préservés lors du passage à la forme canonique. Les sommets correspondants aux états, aux lettres et au graphe en hélice simple représentent trois classes (correspondant aux couleurs sur la figure \ref{fig:helix-aug}) qui ne doivent pas être mélangées les unes avec les autres par \textrm{Nauty}.

\subsubsection*{Quelques optimisations conséquentes}
Nous avions commencé par implémenter la génération en \textrm{Python}, mais nous avions deux inconvénients majeurs :
\begin{itemize}
\item La complexité en temps.
\item \textrm{Nauty} ne dispose pas d'interface efficace en Python.
\end{itemize}

Nous avons donc décidé d'implémenter une version C beaucoup plus efficace. Nous avons également pu augmenter l'efficacité de notre programme en le parallélisant.

\subsubsection*{Quelques benchmarks}

Les tests ont été réalisés sur une machine à 40 cœurs physiques, 80 virtuels, avec à chaque fois un nombre de processus parallèles égal à $|\mathcal{Q}|\times|\Sigma|$.

\begin{table}[h!]
  \begin{center}
    \begin{threeparttable}
      \begin{tabular}{|rrrr|}
        \hline
        \#états & \#lettres & Temps d'exécution & \#biréversibles \\ [0.5ex]
        \hline\hline
        2 & 2 & 0.002s & 12 \\
        \hline
        3 & 2 & 0.003s & 144 \\
        \hline
        3 & 3 & 0.011s & 8 784 \\
        \hline
        4 & 3 & 0.156s & 1 092 096 \\
        \hline
        5 & 3 & 21s    & 16 128 000 \\
        \hline
        4 & 4 & 1m51s  & 1 031 000 000 \\
        \hline
        6 & 3 & 145m44s& 9 848 143 872 \\
        \hline
      \end{tabular}

      \caption{Benchmark génération d'automates biréversibles}
    \end{threeparttable}
  \end{center}
\end{table}


\begin{table}[h!]
  \begin{center}
    \begin{threeparttable}
      \begin{tabular}{|rrrr|}
        \hline
        \#états & \#lettres & Temps d'exécution & \#biréversibles à isomorphisme près\\ [0.5ex]
        \hline\hline
        2 & 2 & 0.002s & 8 \\
        \hline
        3 & 2 & 0.003s & 28 \\
        \hline
        3 & 3 & 0.011s & 335  \\
        \hline
        4 & 3 & 2.59s & 8 605 \\
        \hline
        5 & 3 & 12m34s    & 347 752 \\
        \hline
        4 & 4 & 56min34s  & 1 831 488 \\
        \hline
        6 & 3 & $\infty$  &  \\
        \hline
      \end{tabular}

      \caption{Benchmark génération d'automates biréversibles à isomorphisme près\label{table:birev-iso}}
    \end{threeparttable}
  \end{center}
\end{table}

On constate que Nauty ralentit très significativement le temps d'exécution.

\subsection{Algorithme de factorisation\label{sec:facto}}

  Notons $\mathcal{Q}$ les états de $\mathcal{A}$, $\mathcal{Q'}$ les états de $\mathcal{B}$ et $\mathcal{Q''}$ les états de $\mathcal{M}$. On remarque que tel qu'a été défini le produit (définition \ref{def:produit}) les états de $\mathcal{AB}$ sont des éléments de $\mathcal{Q}\times\mathcal{Q'}$.

Or, quand on factorise un automate, l'ensemble de ses états n'apparaît pas comme un produit d'ensemble. Aussi faut-il choisir une manière canonique de construire une bijection $\iota$ de $\mathcal{Q''}$ dans $\mathcal{Q}\times\mathcal{Q'}$ qui représente cette correspondance. Il est clair que si une telle bijection n'existe pas -- c'est-à-dire si $|\mathcal{Q}''| \ne |\mathcal{Q}|\times|\mathcal{Q'}|$ -- l'automate $\mathcal{M}$ n'est pas factorisable en $\mathcal{AB}$.


On remarquera que toutes les bijections $\iota$ ne conduisent pas à une factorisation. Pour que l'automate soit factorisable, il doit seulement en exister une. On remarque aussi que si on fixe la bijection $\iota$ il existe au moins une permutation $\sigma_\iota\in\mathfrak{S}(\mathcal{Q})$ des états de l'automate tel que $\left(\sigma(\mathcal{Q}), \Sigma, \delta, \rho\right)$ soit factorisable selon la bijection $\iota$.

\begin{figure}[h!]
  \begin{subfigure}[b]{0.5\textwidth}
    \centering
    \begin{tikzpicture}
      \draw [->,>=latex] (-0.5, 0) node[left] {$p$} to (0.5, 0) node[right] {$q$};
      \draw [->,>=latex] (0, 0.5) node[above] {$x$} to (0, -0.5) node[below] {$y$};
    \end{tikzpicture}
    \caption{Une flèche de l'automate $\mathcal{M}$}
  \end{subfigure}
  ~
  \begin{subfigure}[b]{0.5\textwidth}
    \centering
    \begin{tikzpicture}
      \draw [->,>=latex] (-0.5, 1) node[left] {$p_0$} to (0.5, 1) node[right] {$q_0$};
      \draw [->,>=latex] (0, 1.5) node[above] {$x$} to (0, 0.5);

      \draw [->,>=latex] (-0.5, -0.5) node[left] {$p_1$} to (0.5, -0.5) node[right] {$q_1$};
      \draw [->,>=latex] (0, 0) node[above] {$z$} to (0, -1) node[below] {$y$};
    \end{tikzpicture}
    \caption{En haut l'automate $\mathcal{A}$ et en bas $\mathcal{B}$\label{fig:factor-ab}}
  \end{subfigure}
  \caption{Correspondance entre les flèches dans $\mathcal{M}$ et celles de $\mathcal{A}$ et $\mathcal{B}$. Ici on a que $\iota(p) = (p_0,~p_1)$ et $\iota(q)=(q_0, q_1)$\label{fig:facto}.}
\end{figure}

On remarque alors, que connaissant $\mathcal{M}$, la seule donnée inconnue de la figure \ref{fig:factor-ab} est le $z$. Notre algorithme de factorisation consiste donc à essayer tous les $z$ possibles pour chaque arête de $\mathcal{M}$ pour chaque couple de diviseurs du nombre d'états de $\mathcal{M}$. On construit ainsi toutes les flèches des facteurs.

On peut borner grossièrement la complexité en temps de l'algorithme par $\mathcal{O}(|\mathcal{Q''}|!\left(|\mathcal{Q''}||\Sigma|\right)^2)$.

\section{Avancées sur le problème de finitude\label{sec:finitude}}

\subsection{Décider de la finitude du groupe engendré}

Le problème est indécidable en général mais pour tester notre conjecture nous avons besoin d'être capable de savoir si un groupe engendre du fini. Un critère simple pour étudier la finitude d'un automate est d'étudier sa fonction de croissance.

\begin{definition}
  Soit $\mathcal{A}$ une machine de Mealy. On appelle \textbf{\textit{sa fonction de croissance}} la fonction $\pi:\mathbb{N}\rightarrow\mathbb{N}$ qui à un $n$ associe le nombre d'états de $\mathfrak{m}\left(\mathcal{A}^n\right)$.
\end{definition}

\begin{prop}
  \label{prop:mass}
  Si la fonction de masse d'une machine de Mealy $\mathcal{A}$ est bornée, alors $\left<\mathcal{A}\right>_+$ est fini.
\end{prop}

La proposition \ref{prop:mass} nous donne donc un critère de semi-décidabilité sur la finitude du (semi-)groupe engendré par l'automate. En effet, pour décider de la finitude du groupe engendré par $\mathcal{A}$ on peut calculer le nombre d'états successifs de $\mathfrak{m}\left(\mathcal{A}^n\right)$ jusqu'à ce que ce nombre de stabilise. Cependant, si l'automate engendre de l'infini la procédure ne terminera pas. De plus la minimisation et le produit d'automate sont des opérations coûteuses (de complexités respectives $\mathcal{O}\left(\Sigma\mathcal{Q}\log\mathcal{Q}\right)$ et $\mathcal{O}\left(\mathcal{Q}^2\Sigma\right)$) qui peuvent être à répéter de nombreuses fois.

\subsection{$\mathfrak{md}$-triviaux dans des classes d'automates irréductibles}

Comme dit dans la section \ref{sec:action} la $\mathfrak{md}$-réduction résout le problème de finitude pour les automates biréversibles à deux états/lettres. La figure \ref{fig:fantastiques} montre les quatre automates qui ne se $\mathfrak{md}$-trivialisent pas mais engendrent du fini pour les biréversibles à 4 états et 3 lettres. Ces automates se factorisent, et en permutant l'ordre de leurs facteurs on trouve des automates $\mathfrak{md}$-triviaux.

Une première manière de tester la conjecture \ref{conj:birev-mdc} est de tester si la $\mathfrak{md}$-réduction suffit pour décider de la finitude dans des cas où la factorisation est impossible -- \emph{i.e.} quand le nombre d'états est premier.

Nous avons donc procédé de la manière suivante:

\begin{enumerate}
\item engendrer une classe d'automates biréversibles irréductibles à isomorphismes près
\item calculer la borne supérieure des fonctions de croissance des automates $\mathfrak{md}$-triviaux
\item vérifier que la fonction de croissance de tous les automates $\mathfrak{md}$-trivaux dépasse cette borne.
\end{enumerate}

Le calcul sur les biréversibles à 3 états et 3 lettres est très rapide et ne montre pas de contres exemples. La plus petite classe d'automates biréversibles irréductibles est ensuite les automates à 5 états et 3 lettres. Nous en avions dénombrés 347 752 à isomorphisme près (table \ref{table:birev-iso}). Pour arriver au bout de cette classe d'automates, nous avons implémenté le calcul du $\mathfrak{md}$-réduit et de la fonction de croissance en C puis paralléliser son exécution. Grâce au super calculateur de l'université nous avons pu trouver 8 contres exemples.

\begin{figure}[!ht]
  \begin{center}
    \scalebox{0.8}{
  \begin{tikzpicture}
    \coordinate (A) at (-5, 0);
    \coordinate (B) at (-2, -2);
    \coordinate (C) at (2, 2);
    \coordinate (D) at (-2, 2);
    \coordinate (E) at (2, -2);

    \node[draw,circle,minimum width = 2.5em] (QA) at (A) {$a$};
    \node[draw,circle,minimum width = 2.5em] (QB) at (B) {$b$};
    \node[draw,circle,minimum width = 2.5em] (QC) at (C) {$c$};
    \node[draw,circle,minimum width = 2.5em] (QD) at (D) {$d$};
    \node[draw,circle,minimum width = 2.5em] (QE) at (E) {$e$};

    \draw[->,>=latex] (QA) to[out=70,in=110,looseness=6] node[midway, above]{\begin{tabular}{c} $0|0$ \\ $1|2$ \\ $2|1$ \end{tabular}} (QA);
    \draw[->,>=latex] (QB) to[out=205,in=245,looseness=6] node[midway, below left] {$2|0$} (QB);
    \draw[->,>=latex] (QB) to[bend left] node[midway, left] {$0|2$} (QD);
    \draw[->,>=latex] (QB) to[bend left] node[midway, above] {$1|1$} (QE);
    \draw[->,>=latex] (QC) to[out=25,in=65,looseness=6] node[midway, above right]{$1|0$} (QC);
    \draw[->,>=latex] (QC) to[bend left] node[midway, below] {$2|1$} (QD);
    \draw[->,>=latex] (QC) to[bend left] node[midway, right] {$0|2$} (QE);
    \draw[->,>=latex] (QD) to[out=115,in=155,looseness=6] node[midway, above left]{$1|0$} (QD);
    \draw[->,>=latex] (QD) to[bend left] node[midway, right] {$0|1$} (QB);
    \draw[->,>=latex] (QD) to[bend left] node[midway, above] {$2|2$} (QC);
    \draw[->,>=latex] (QE) to[out=295,in=335,looseness=6] node[midway, below right]{$2|0$} (QE);
    \draw[->,>=latex] (QE) to[bend left] node[midway, below] {$1|2$} (QB);
    \draw[->,>=latex] (QE) to[bend left] node[midway, left] {$0|1$} (QC);
  \end{tikzpicture}
  }
  \caption{Contre exemple à la conjecture \ref{conj:birev-mdc}.\label{fig:contre-exemple}}
\end{center}
\end{figure}

\subsection{Automates fantastiques}

Nous avons ensuite cherché dans des classes plus grandes que celles abordées jusqu'alors pour trouver des automates \textit{fantastiques}.

\begin{definition}
  Un automate est \textbf{\textit{fantastique}} s'il est $\mathfrak{mdc}$-trivial mais pas $\mathfrak{md}$-trivial.
\end{definition}

La factorisation est une opération coûteuse, pour tester la $\mathfrak{mdc}$-réduction il est donc préférable d'énumérer les classes de diviseurs et de faire tous les produits. On trouve ainsi tous les automates fantastiques d'une classe à isomorphisme près.

\begin{table}[h!]
  \begin{center}
    \begin{threeparttable}
      \begin{tabular}{|rrr|}
        \hline
        \#états & \#lettres & \#fantastiques \\ [0.5ex]
        \hline\hline
        \hline
        4 & 3 & 4 \\
        \hline
        4 & 4 & 40 \\
        \hline
        6 & 3 & 0 \\
        \hline
      \end{tabular}
      \caption{Nombre d'automates fantastiques pour des classes d'automates biréversibles}
    \end{threeparttable}
  \end{center}
\end{table}

\section{Conclusions}

La $\mathfrak{mdc}$-réduction ne fournit pas un critère suffisant pour résoudre le problème de finitude des groupes engendrés par des automates biréversibles (contre exemple figure \ref{fig:contre-exemple}).

On pourrait se demander s'il y a des points communs entre les fantastiques et les contre-exemples irréductibles que nous avons trouvé. Nous pourrions alors peut-être ajouter une opération à la $\mathfrak{mdc}$-réduction qui résolve le problème de finitude pour les groupes d'automates biréversibles.

Les contre-exemples que nous avons trouvé fournissent une nouvelle source d'information sur les limites de la $\mathfrak{md}$-réduction. Aussi, permetteront'il peut être d'envisager le problème de finitude sous de nouvelles perspectives.


\newpage
\bibliography{project}{
  \nocite{*}
}
\bibliographystyle{plain}

\end{document}
