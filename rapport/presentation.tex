\documentclass[11pt]{beamer}
\usepackage[utf8]{inputenc}
\usepackage[T1]{fontenc}
\usepackage{lmodern}
\usetheme{metropolis}

\usepackage{amsmath}
\usepackage{amssymb}
\usepackage{amsthm}
\usepackage{mathtools}
\usepackage{caption}
\usepackage{xfrac}
\usepackage{graphicx}
\usepackage{subcaption}
\usepackage{listings}
\usepackage{fancyvrb}
\usepackage{threeparttable}

\usepackage{tikz}
\usetikzlibrary{overlay-beamer-styles,calc,automata,positioning}

\usepackage[french]{babel}

\newtheorem{prop}{Proposition}
\newtheorem{defi}{Définition}
\newtheorem{thm}{Théorème}
\newtheorem{conj}{Conjecture}
\newtheorem*{rem*}{Remarque}

\begin{document}
\author{Maxime Flin  \& Tristan François}
\title{\`A propos du problème de finitude des groupes d'automates biréversibles}
% \subtitle{}
% \logo{}
% \institute{}
% \date{}
% \subject{}
% \setbeamercovered{transparent}
% \setbeamertemplate{navigation symbols}{}
\begin{frame}[plain]
  \maketitle
\end{frame}

\begin{frame}
  \frametitle{Sommaire}
  \tableofcontents
\end{frame}

\section{Automates de Mealy et position du problème}

\begin{frame}
    \frametitle{Automate de Mealy : définition}

    \begin{defi}
        Un \textbf{automate de Mealy} est la donné d'un quadruplet $\left(\mathcal{Q}, \Sigma, \delta, \rho\right)$ où $\mathcal{Q}$ est l'ensemble des états de l'automate, $\Sigma$ l'alphabet sur lequel l'automate agit, $\delta$ une famille d'applications de $\mathcal{Q}$ dans $\mathcal{Q}$ indexées par $\Sigma$ qui représente les transitions entre les états et $\rho$ une famille d'applications de $\Sigma$ dans $\Sigma$ indexées par $\mathcal{Q}$ qui représente l'écriture en sortie de l'automate.
    \end{defi}

\end{frame}

\begin{frame}
  \frametitle{Automate de Mealy : exemple}

  \begin{figure}
    \centering
    \begin{tikzpicture}
      \coordinate (A) at (0,0);
      \coordinate (B) at (3,0);
      \coordinate (C) at (6,0);

      \node[draw,circle,minimum width = 2.5em,alt={<3>{orange}{black}}] (QA) at (A) {$0$};
      \node[draw,circle,minimum width = 2.5em,alt={<5,11>{orange}{black}}] (QB) at (B) {$1$};
      \node[draw,circle,minimum width = 2.5em,alt={<7,9>{orange}{black}}] (QC) at (C) {$2$};

      \draw[->,>=latex] (QA) to[out=160,in=200,looseness=6] node[midway, left] {$0|0$}(QA);
      \draw[->,>=latex,alt=<4>{orange}{black}] (QA) to[bend left] node[midway, above] {$1|0$} (QB);
      \draw[->,>=latex] (QB) to[bend left] node[midway, below] {$1|1$}(QA);
      \draw[->,>=latex, alt={<6>{orange}{black}}] (QB) to[bend left] node[midway, above] {$0|0$}(QC);
      \draw[->,>=latex,alt={<10>{orange}{black}}] (QC) to[bend left] node[midway, below] {$0|1$}(QB);
      \draw[->,>=latex,alt={<8>{orange}{black}}] (QC) to[out=340,in=20,looseness=6] node[midway, right] {$1|1$}(QC);

      \pause

      \coordinate (C1) at (0,-2);
      \coordinate (C2) at (2,-2);
      \coordinate (C3) at (4,-2);
      \coordinate (C4) at (6,-2);

      \node[alt={<3>{orange}{black}}] (CC1) at (C1) {$1$};
      \node[alt={<5>{orange}{black}}] (CC2) at (C2) {$0$};
      \node[alt={<7>{orange}{black}}] (CC3) at (C3) {$1$};
      \node[] (CC4) at (C4) {$0$};

      \pause

      \coordinate (E1) at (-1,-3);

      \node[alt={<3>{orange}{black}}] (EE1) at (E1) {$0$};

      \pause

      \coordinate (E2) at (1,-3);
      \coordinate (O1) at (0,-4);

      \node[alt={<4,5>{orange}{black}}] (EE2) at (E2) {$1$};
      \node[alt={<4,12>{orange}{black}}] (OO1) at (O1) {$0$};

      \draw[->,>=latex,alt={<4>{orange}{black}}] (EE1) to (EE2);
      \draw[->,>=latex,alt={<4>{orange}{black}}] (CC1) to (OO1);

      \pause
      \pause

      \coordinate (E3) at (3,-3);
      \coordinate (O2) at (2,-4);

      \node[alt={<6,7>{orange}{black}}] (EE3) at (E3) {$2$};
      \node[alt={<6,12>{orange}{black}}] (OO2) at (O2) {$0$};

      \draw[->,>=latex,alt={<6>{orange}{black}}] (EE2) to (EE3);
      \draw[->,>=latex,alt={<6>{orange}{black}}] (CC2) to (OO2);

      \pause
      \pause

      \coordinate (E4) at (5,-3);
      \coordinate (O3) at (4,-4);

      \node[alt={<8,9>{orange}{black}}] (EE4) at (E4) {$2$};
      \node[alt={<8,12>{orange}{black}}] (OO3) at (O3) {$1$};

      \draw[->,>=latex,alt={<8>{orange}{black}}] (EE3) to (EE4);
      \draw[->,>=latex,alt={<8>{orange}{black}}] (CC3) to (OO3);

      \pause

      \coordinate (E5) at (7,-3);
      \coordinate (O4) at (6,-4);

      \node[alt={<10,11>{orange}{black}}] (EE5) at (E5) {$1$};
      \node[alt={<10,12>{orange}{black}}] (OO4) at (O4) {$1$};

      \draw[->,>=latex,alt={<10>{orange}{black}}] (EE4) to (EE5);
      \draw[->,>=latex,alt={<10>{orange}{black}}] (CC4) to (OO4);
    \end{tikzpicture}
    \caption{Automate de Mealy réalisant la division par 3 en binaire.}
  \end{figure}
\end{frame}

\begin{frame}
  \frametitle{Automate inversible, réversible et biréversible~i}

  \begin{defi}
    Soit $\mathcal{A}=\left(\mathcal{Q}, \Sigma, \delta, \rho\right)$ un automate de Mealy.
    \begin{enumerate}[(i)]
    \item On dit que $\mathcal{A}$ est \textbf{inversible} si $\rho_q$ est une permutation de $\Sigma$ pour tout état $q\in\mathcal{Q}$.
    \item On dit que $\mathcal{A}$ est \textbf{réversible} si $\delta_x$ est une permutation de $\mathcal{Q}$ pour toute lettre $x\in\Sigma$.
    \end{enumerate}
  \end{defi}

  \begin{figure}[!ht]
    \begin{subfigure}[b]{0.4\textwidth}
      \centering
      \begin{tabular}{|r|r|r|}
        \hline
        1 & 2 & 2 \\
        \hline
        2 & 0 & 1 \\
        \hline
        0 & 1 & 0 \\
        \hline
      \end{tabular}
      \caption{$\delta$ d'un automate réversible.}
    \end{subfigure}
    ~
    \begin{subfigure}[b]{0.4\textwidth}
      \centering
      \begin{tabular}{|r|r|r|}
        \hline
        0 & 1 & 2 \\
        \hline
        2 & 0 & 1 \\
        \hline
        0 & 1 & 2 \\
        \hline
      \end{tabular}
      \caption{$\rho$ d'un automate inversible.}
    \end{subfigure}
  \end{figure}


\end{frame}

\begin{frame}
  \frametitle{Automate inversible, réversible et biréversible~ii}

  \begin{defi}
    Si une machine de Mealy $\mathcal{M}$ est inversible, alors on peut construire son \textbf{\textit{inverse}}, noté $\mathcal{M}^{-1}$, en remplaçant chaque flèche $p\overset{x|y}{\longrightarrow}q$ par $p^{-1}\overset{y|x}{\longrightarrow}q^{-1}$.
  \end{defi}


  \begin{figure}[h!]
    \centering
    \scalebox{.9} {
      \begin{subfigure}[b]{0.5\textwidth}
        \centering
        \begin{tikzpicture}[shorten >=1pt,node distance=2cm,on grid,auto]
          \node[state] (p) {$p$};
          \node[state] (q) [right=of p] {$q$};
          \path[->,>=latex]
          (p) edge [loop left] node {$1|0$} (p)
          (p) edge [bend left] node {$0|1$} (q)
          (q) edge [bend left] node {$0|0$} (p)
          (q) edge [loop right] node {$1|1$} (q);
        \end{tikzpicture}
      \end{subfigure}
      ~
      \begin{subfigure}[b]{0.5\textwidth}
        \centering
        \begin{tikzpicture}[shorten >=1pt,node distance=2cm,on grid,auto]
          \node[state] (p) {$p^{-1}$};
          \node[state] (q) [right=of p] {$q^{-1}$};
          \path[->,>=latex]
          (p) edge [loop left] node {$0|1$} (p)
          (p) edge [bend left] node {$1|0$} (q)
          (q) edge [bend left] node {$0|0$} (p)
          (q) edge [loop right] node {$1|1$} (q);
        \end{tikzpicture}
      \end{subfigure}
    }
    \caption{Automate et son inverse.}
  \end{figure}
\end{frame}

\begin{frame}
  \frametitle{Automate inversible, réversible et biréversible~iii}

  \begin{defi}
    Si une machine de Mealy est réversible, alors on peut construire son \textbf{\textit{dual}}, noté $\mathfrak{d}\mathcal{M}$, en remplacant chaque flèche $p\overset{x|y}{\longrightarrow}q$ par $x\overset{p|q}{\longrightarrow}y$.
  \end{defi}


  \begin{figure}[h!]
    \centering
    \scalebox{.9}{
      \begin{subfigure}[b]{0.5\textwidth}
        \centering
        \begin{tikzpicture}[shorten >=1pt,node distance=2cm,on grid,auto]
          \node[state] (p) {$p$};
          \node[state] (q) [right=of p] {$q$};
          \path[->,>=latex]
          (p) edge [loop left] node {$1|0$} (p)
          (p) edge [bend left] node {$0|1$} (q)
          (q) edge [bend left] node {$0|0$} (p)
          (q) edge [loop right] node {$1|1$} (q);
        \end{tikzpicture}
      \end{subfigure}
      ~
      \begin{subfigure}[b]{0.5\textwidth}
        \centering
        \begin{tikzpicture}[shorten >=1pt,node distance=2cm,on grid,auto]
          \node[state] (p) {$0$};
          \node[state] (q) [right=of p] {$1$};
          \path[->,>=latex]
          (p) edge [loop left] node {$q|p$} (p)
          (p) edge [bend left] node {$p|q$} (q)
          (q) edge [bend left] node {$p|p$} (p)
          (q) edge [loop right] node {$q|q$} (q);
        \end{tikzpicture}
      \end{subfigure}
    }
    \caption{Automate et son dual}
  \end{figure}
\end{frame}

\begin{frame}
  \frametitle{Automate inversible, réversible et biréversible~iv}

  \begin{defi}
    On dit que $\mathcal{A}$ est \textbf{biréversible} s'il est inversible, réversible et que son inverse est réversible.
  \end{defi}

  C'est à cette classe d'automates en particulier que nous allons nous intéresser.
\end{frame}

\begin{frame}
  \frametitle{Produit d'automates~i}
  \begin{defi}
    Soient $\mathcal{A} = \left(\mathcal{Q}, \Sigma, \delta, \rho\right)$ et $\mathcal{B} = \left(\mathcal{Q'}, \Sigma, \delta', \rho'\right)$ deux automates sur le même alphabet. On appelle \textbf{\textit{automate produit}} la machine $\mathcal{AB} = \left(\mathcal{Q}\times\mathcal{Q'}, \Sigma, \delta'', \rho''\right)$ avec

    \[ \delta_x''(p_0, p_1) = (\delta_x(p), \delta_{\rho_{p_0}(x)}'(q))\]
    et
    \[ \rho_{(p_0,p_1)}(x) = \rho_{p_1}'(\rho_{p_0}(x)). \]
  \end{defi}
\end{frame}

\begin{frame}
  \frametitle{Produit d'automates~ii}
  \begin{figure}[h!]
    \begin{subfigure}[b]{0.3\textwidth}
      \centering
      \begin{tikzpicture}
        \draw [->,>=latex] (-0.5, 1) node[left] {$p_0$} to (0.5, 1) node[right] {$q_0$};
        \draw [->,>=latex] (0, 1.5) node[above] {$x$} to (0, 0.5);

        \draw [->,>=latex] (-0.5, -0.5) node[left] {$p_1$} to (0.5, -0.5) node[right] {$q_1$};
        \draw [->,>=latex] (0, 0) node[above] {$z$} to (0, -1) node[below] {$y$};
      \end{tikzpicture}
      \caption{En haut l'automate $\mathcal{A}$ et en bas $\mathcal{B}$.}
    \end{subfigure}
    ~
    \begin{subfigure}[b]{0.3\textwidth}
      \centering
      \begin{tikzpicture}
        \draw [->,>=latex] (-0.5, 0) node[left] {$p_0p_1$} to (0.5, 0) node[right] {$q_0q_1$};
        \draw [->,>=latex] (0, 0.5) node[above] {$x$} to (0, -0.5) node[below] {$y$};
      \end{tikzpicture}
      \caption{Une flèche de l'automate $\mathcal{AB}$.}
    \end{subfigure}
    \caption{Correspondance entre les flèches dans $\mathcal{AB}$ et celles de $\mathcal{A}$ et de $\mathcal{B}$.}
  \end{figure}
\end{frame}

\begin{frame}
  \frametitle{(Semi-)groupe engendré par un automate de Mealy~i}

  On peut étendre l'ensemble de définition de $\rho$ à $\Sigma^*$ par induction comme suit.

  Soient $p\in\mathcal{Q}$, $x\in\Sigma$ et $\textbf{u}\in\Sigma^*$
  \begin{align*}
    &\rho_p(\varepsilon)=\varepsilon \\
    &\rho_p(x\textbf{u})=\rho_p(x)\rho_{\delta_x(p)}(\textbf{u})
  \end{align*}


  \begin{figure}[!ht]
    \begin{center}
      \begin{tikzpicture}
        \coordinate (P1) at (0.5,0);
        \coordinate (P2) at (2.5,0);
        \coordinate (Q1) at (5,0);

        \coordinate (00) at (1.25,1);
        \coordinate (01) at (3.60,1);

        \coordinate (10) at (1.25,-1);
        \coordinate (11) at (3.60,-1);

        \node (PP1) at (P1) {$p$};
        \node (PP2) at (P2) {$\delta_x(p)$};
        \node (QQ1) at (Q1) {$\delta_{\textbf{u}}(\delta_x(p))$};

        \node (L00) at (00) {$x$};
        \node (L01) at (01) {$\textbf{u}$};

        \node (L10) at (10) {$\rho_p(x)$};
        \node (L11) at (11) {$\rho_{\delta_x(p)}(\textbf{u})$};

        \draw[->,>=latex] (PP1) -- (PP2);
        \draw[->,>=latex] (PP2) -- (QQ1);

        \draw[->,>=latex] (L00) -- (L10);
        \draw[->,>=latex] (L01) -- (L11);
      \end{tikzpicture}
    \end{center}
  \end{figure}
\end{frame}

\begin{frame}
  \frametitle{(Semi-)groupe engendré par un automate de Mealy~ii}

  \begin{defi}
    On définit le \textbf{semi-groupe d'automate engendré par $\mathcal{A}$} comme
    \begin{equation*}
      \left<\mathcal{A}\right>_+=\left<\rho_p, \forall p\in\mathcal{Q}\right>
    \end{equation*}

    Si l'automate $\mathcal{A}$ est inversible, alors on peut aussi considérer l'inverse des $\rho_p$. Dans ce cas, l'automate engendre un groupe noté $\left<\mathcal{A}\right>$.

  \end{defi}
\end{frame}

\begin{frame}
  \frametitle{Minimisation et Dualisation : $\mathfrak{md}$-réduction}

    \begin{defi}
        Soit $\mathcal{A}=\left(\mathcal{Q}, \Sigma, \delta, \rho\right)$ un automate de Mealy. \textbf{L'équivalence de Nérode} $\equiv$ sur $\mathcal{Q}$ est la limite de la suite de relations d'équivalence $\equiv_k$ de plus en plus fines définies récursivement par
        \begin{align*}
        \forall p, q \in Q,\\
        p \equiv_0 q &\iff \rho_p = \rho_q \\
        \forall k \geq 0, p \equiv_{k+1}q &\iff \left(p\equiv_kq \wedge \forall x \in \Sigma,~\delta_x(p)=\delta_x(q)\right).
        \end{align*}

        Le \textbf{minimisé} de $\mathcal{A}$ est l'automate $\mathfrak{m}(\mathcal{A})=\left(\sfrac{\mathcal{Q}}{\equiv}, \Sigma, \bar{\delta}, \bar{\rho}\right)$, où pour tout $p\in\mathcal{Q}$ et $x\in\Sigma$, $\bar{\delta}_x([p]) = [\delta_x(p)]$ et $\bar{\rho}_{[p]} =  \rho_p$.
    \end{defi}

    Akhavi \emph{et al}~\cite{DBLP:journals/corr/abs-1105-4725} ont proposé la $\mathfrak{md}$-réduction, une procédure alternant minimisation et dualisation de l'automate.
\end{frame}

\begin{frame}
    \frametitle{Exemple de $\mathfrak{md}$-réduction}
    \begin{figure}[!ht]
        %\centering
        \scalebox{0.5}{
            \begin{tikzpicture}
            \coordinate (A2) at (-2,-7);
            \coordinate (B2) at (-2,-11);
            \coordinate (C2) at (2,-7);
            \coordinate (D2) at (2,-11);
            \node[draw,thick,circle,minimum width = 2.5em] (QA2) at (A2) {$b$};
            \node[draw,thick,circle,minimum width = 2.5em] (QB2) at (B2) {$a$};
            \node[draw,thick,circle,minimum width = 2.5em] (QC2) at (C2) {$c$};
            \node[draw,thick,circle,minimum width = 2.5em] (QD2) at (D2) {$d$};
            \draw[thick,->,>=latex] (QA2) to[out=115,in=155,looseness=6] node[midway, above left]{$1|2$} (QA2);
            \draw[thick,->,>=latex] (QA2) to[bend left] node[midway, above]{\begin{tabular}{c} $0|1$ \\ $2|0$ \end{tabular}} (QC2);
            \draw[thick,->,>=latex] (QB2) to[out=205,in=245,looseness=6] node[midway, below left]{$1|2$} (QB2);
            \draw[thick,->,>=latex] (QB2) to[bend left] node[midway, above]{\begin{tabular}{c} $0|0$ \\ $2|1$ \end{tabular}} (QD2);
            \draw[thick,->,>=latex] (QC2) to[out=25,in=65,looseness=6] node[midway, above right]{$1|2$} (QC2);
            \draw[thick,->,>=latex] (QC2) to[bend left] node[midway, below]{\begin{tabular}{c} $0|1$ \\ $2|0$ \end{tabular}} (QA2);
            \draw[thick,->,>=latex] (QD2) to[out=295,in=335,looseness=6] node[midway, below right]{$1|2$} (QD2);
            \draw[thick,->,>=latex] (QD2) to[bend left] node[midway, below]{\begin{tabular}{c} $0|0$ \\ $2|1$ \end{tabular}} (QB2);

            % minization
            \draw[thick,->,>=latex] (4, -9) to node[midway, above] {$\mathfrak{m}$} (5, -9);

            % minimisé du conjugué

            \coordinate (A2) at (7,-10);
            \coordinate (B2) at (10,-10);
            \node[draw,thick,circle,minimum width = 2.5em] (QA2) at (A2) {$bc$};
            \node[draw,thick,circle,minimum width = 2.5em] (QB2) at (B2) {$ad$};
            \draw[thick,->,>=latex] (QA2) to[out=70,in=110,looseness=6] node[midway, above]{\begin{tabular}{c} $0|0$ \\ $1|2$ \\ $2|1$ \end{tabular}} (QA2);
            \draw[thick,->,>=latex] (QB2) to[out=70,in=110,looseness=6] node[midway, above]{\begin{tabular}{c} $0|1$ \\ $1|2$ \\ $2|0$ \end{tabular}} (QB2);

            % dualisation

            \draw[thick,<->,>=latex] (12, -9) to node[midway, above] {$\mathfrak{d}$} (13, -9);

            \coordinate (A2) at (15,-10);
            \coordinate (B2) at (18,-10);
            \coordinate (C2) at ($(A2) + (60:3)$);
            \node[draw,thick,circle,minimum width = 2.5em] (QA2) at (A2) {$0$};
            \node[draw,thick,circle,minimum width = 2.5em] (QB2) at (B2) {$2$};
            \node[draw,thick,circle,minimum width = 2.5em] (QC2) at (C2) {$1$};

            \draw[thick,->,>=latex] (QA2) to[out=190,in=230,looseness=6] node[midway, below left]{$ad|ad$} (QA2);
            \draw[thick,->,>=latex] (QA2) to node[midway, above left] {$bc|bc$} (QC2);
            \draw[thick,->,>=latex] (QB2) to node[midway, below] {$bc|bc$} (QA2);
            \draw[thick,->,>=latex] (QB2) to[bend left] node[midway, below left] {$ad|ad$} (QC2);
            \draw[thick,->,>=latex] (QC2) to[bend left] node[midway, above right] {\begin{tabular}{c} $ad|ad$ \\ $bc|bc$ \end{tabular}} (QB2);

            %minimisation
            \draw[thick,->,>=latex] (16.5, -12) to node[midway, right] {$\mathfrak{m}$} (16.5, -13);

            \coordinate (A2) at (16.5, -16);
            \node[draw,thick,circle,minimum width = 2.5em] (QA2) at (A2) {$012$};
            \draw[thick,->,>=latex] (QA2) to[out=70,in=110,looseness=6] node[midway, above]{\begin{tabular}{c} $ad|ad$ \\ $bc|bc$ \end{tabular}} (QA2);

            %dualisation

            \draw[thick,<->,>=latex] (12, -15) to node[midway, above] {$\mathfrak{d}$} (13, -15);

            \coordinate (A2) at (7, -15.5);
            \coordinate (B2) at (10, -15.5);
            \node[draw,thick,circle,minimum width = 2.5em] (QA2) at (A2) {$ad$};
            \node[draw,thick,circle,minimum width = 2.5em] (QB2) at (B2) {$bc$};
            \draw[thick,->,>=latex] (QA2) to[out=70,in=110,looseness=6] node[midway, above]{$012|012$} (QA2);
            \draw[thick,->,>=latex] (QB2) to[out=70,in=110,looseness=6] node[midway, above]{$012|012$} (QB2);

            %minimisation
            \draw[thick,->,>=latex] (5, -15) to node[midway, above] {$\mathfrak{m}$} (4, -15);

            \coordinate (A2) at (0, -15.5);
            \node[draw,thick,circle,minimum width = 2.5em] (QA2) at (A2) {$adbc$};
            \draw[thick,->,>=latex] (QA2) to[out=70,in=110,looseness=6] node[midway, above]{$012|012$} (QA2);

            \end{tikzpicture}
        }
        \caption{$\mathfrak{md}$-réduction de l'automate $\mathcal{B}$.\label{fig:md-red-ex}}
    \end{figure}
\end{frame}

\begin{frame}
  \frametitle{Objectif du Projet~i}
  \begin{thm}[Klimann~\cite{Klimann13}]
    Tout automate inversible-réversible à deux lettres et/ou deux états engendre un groupe fini si et seulement s'il est $\mathfrak{md}$-trivial.
  \end{thm}

  Le but du projet était d'essayer d'étendre cette conjecture à des classes d'automates plus grandes.
\end{frame}

\begin{frame}
  \frametitle{Des contre-exemples au théorème dans le cas général}

  \begin{figure}[!ht]
    \scalebox{.5}{
      \centering
      \begin{tikzpicture}[scale=0.9]
        \coordinate (A) at (-10,-2);
        \coordinate (B) at (-6,-2);
        \coordinate (C) at (-6,2);
        \coordinate (D) at (-10,2);
        \node[draw,circle,minimum width = 2.5em] (QA) at (A) {};
        \node[draw,circle,minimum width = 2.5em] (QB) at (B) {};
        \node[draw,circle,minimum width = 2.5em] (QC) at (C) {};
        \node[draw,circle,minimum width = 2.5em] (QD) at (D) {};
        \draw[->,>=latex] (QA) to[out=240,in=210,looseness=8] node[midway, below left]{$1|2$} (QA);
        \draw[->,>=latex] (QA) to[bend right] node[midway, below]{$0|1$} (QB);
        \draw[->,>=latex] (QA) to[bend right] node[midway, right]{$2|0$} (QD);
        \draw[->,>=latex] (QB) to[bend right] node[midway, above]{$0|0$} (QA);
        \draw[->,>=latex] (QB) to[out=330, in=300, looseness=8] node[midway, below right]{$1|2$} (QB);
        \draw[->,>=latex] (QB) to[bend right] node[midway, right]{$2|1$} (QC);
        \draw[->,>=latex] (QC) to[bend right] node[midway, left]{$2|0$} (QB);
        \draw[->,>=latex] (QC) to[out=30, in=60, looseness=8] node[midway, above right]{$0|2$} (QC);
        \draw[->,>=latex] (QC) to[bend right] node[midway, above]{$1|1$} (QD);
        \draw[->,>=latex] (QD) to[bend right] node[midway, left]{$2|1$} (QA);
        \draw[->,>=latex] (QD) to[bend right] node[midway, below]{$1|0$} (QC);
        \draw[->,>=latex] (QD) to[out=120, in=150, looseness=8] node[midway, above left]{$0|2$} (QD);


        \coordinate (A) at (0,0);
        \coordinate (B) at (0:3);
        \coordinate (C) at (120:3);
        \coordinate (D) at (240:3);
        \node[draw,circle,minimum width = 2.5em] (QA) at (A) {};
        \node[draw,circle,minimum width = 2.5em] (QB) at (B) {};
        \node[draw,circle,minimum width = 2.5em] (QC) at (C) {};
        \node[draw,circle,minimum width = 2.5em] (QD) at (D) {};
        \draw[->,>=latex] (QA) to[bend right] node[midway, below]{$1|0$} (QB);
        \draw[->,>=latex] (QB) to[bend right] node[midway, above]{$2|0$} (QA);
        \draw[->,>=latex] (QA) to[bend right] node[midway, above right]{$0|1$} (QC);
        \draw[->,>=latex] (QC) to[bend right] node[midway, below left]{$0|2$} (QA);
        \draw[->,>=latex] (QA) to[bend right] node[midway, above left]{$2|2$} (QD);
        \draw[->,>=latex] (QD) to[bend right] node[midway, below right]{$1|1$} (QA);
        \draw[->,>=latex] (QB) to[bend right] node[midway, below]{$1|2$} (QC);
        \draw[->,>=latex] (QC) to[bend left=40] node[midway, above]{$2|1$} (QB);
        \draw[->,>=latex] (QC) to[bend right] node[midway, right]{$1|0$} (QD);
        \draw[->,>=latex] (QD) to[bend left=40] node[midway, left]{$2|0$} (QC);
        \draw[->,>=latex] (QD) to[bend right] node[midway, above]{$0|2$} (QB);
        \draw[->,>=latex] (QB) to[bend left=40] node[midway, below]{$0|1$} (QD);

        \coordinate (A) at (6,-2);
        \coordinate (B) at (10,-2);
        \coordinate (C) at (10,2);
        \coordinate (D) at (6,2);
        \node[draw,circle,minimum width = 2.5em] (QA) at (A) {};
        \node[draw,circle,minimum width = 2.5em] (QB) at (B) {};
        \node[draw,circle,minimum width = 2.5em] (QC) at (C) {};
        \node[draw,circle,minimum width = 2.5em] (QD) at (D) {};
        \draw[->,>=latex] (QA) to[out=240,in=210,looseness=8] node[midway, below left]{$1|0$} (QA);
        \draw[->,>=latex] (QA) to[bend right] node[midway, below]{$0|2$} (QB);
        \draw[->,>=latex] (QA) to[bend right] node[midway, right]{$2|1$} (QD);
        \draw[->,>=latex] (QB) to[bend right] node[midway, above]{$0|2$} (QA);
        \draw[->,>=latex] (QB) to[out=330, in=300, looseness=8] node[midway, below right]{$1|1$} (QB);
        \draw[->,>=latex] (QB) to[bend right] node[midway, right]{$2|0$} (QC);
        \draw[->,>=latex] (QC) to[bend right] node[midway, left]{$2|0$} (QB);
        \draw[->,>=latex] (QC) to[out=30, in=60, looseness=8] node[midway, above right]{$0|1$} (QC);
        \draw[->,>=latex] (QC) to[bend right] node[midway, above]{$1|2$} (QD);
        \draw[->,>=latex] (QD) to[bend right] node[midway, left]{$2|1$} (QA);
        \draw[->,>=latex] (QD) to[bend right] node[midway, below]{$1|2$} (QC);
        \draw[->,>=latex] (QD) to[out=120, in=150, looseness=8] node[midway, above left]{$0|0$} (QD);
      \end{tikzpicture}
    }
    \caption{Les plus petits contre-exemples à une généralisation du théorème \ref{thm:K}. Le quatrième contre-exemple est l'automate inverse du premier. \label{fig:fantastiques}
    }
  \end{figure}
\end{frame}

\begin{frame}
  \frametitle{Un point commun à ces quatre contre-exemples}
  Ils se factorisent en produit de deux automates et en permutant les facteurs on trouve un automate $\mathfrak{md}$-trivial.

  \begin{figure}[!ht]
    \begin{center}
        \scalebox{0.4}{
            \begin{tikzpicture}

            % Automate de départ : fantastique2

            \coordinate (A1) at (0,0);
            \coordinate (B1) at (0:3);
            \coordinate (C1) at (120:3);
            \coordinate (D1) at (240:3);
            \node[draw,thick,circle,minimum width = 2.5em] (QA1) at (A1) {};
            \node[draw,thick,circle,minimum width = 2.5em] (QB1) at (B1) {};
            \node[draw,thick,circle,minimum width = 2.5em] (QC1) at (C1) {};
            \node[draw,thick,circle,minimum width = 2.5em] (QD1) at (D1) {};
            \draw[thick,->,>=latex] (QA1) to[bend right] node[midway, below]{$1|0$} (QB1);
            \draw[thick,->,>=latex] (QB1) to[bend right] node[midway, above]{$2|0$} (QA1);
            \draw[thick,->,>=latex] (QA1) to[bend right] node[midway, above right]{$0|1$} (QC1);
            \draw[thick,->,>=latex] (QC1) to[bend right] node[midway, below left]{$0|2$} (QA1);
            \draw[thick,->,>=latex] (QA1) to[bend right] node[midway, above left]{$2|2$} (QD1);
            \draw[thick,->,>=latex] (QD1) to[bend right] node[midway, below right]{$1|1$} (QA1);
            \draw[thick,->,>=latex] (QB1) to[bend right] node[midway, below]{$1|2$} (QC1);
            \draw[thick,->,>=latex] (QC1) to[bend left=40] node[midway, above]{$2|1$} (QB1);
            \draw[thick,->,>=latex] (QC1) to[bend right] node[midway, right]{$1|0$} (QD1);
            \draw[thick,->,>=latex] (QD1) to[bend left=40] node[midway, left]{$2|0$} (QC1);
            \draw[thick,->,>=latex] (QD1) to[bend right] node[midway, above]{$0|2$} (QB1);
            \draw[thick,->,>=latex] (QB1) to[bend left=40] node[midway, below]{$0|1$} (QD1);

            % égal

            \draw (4,0) node {$=$};

            % Décomposition en produit

            \coordinate (A2) at (6,2);
            \coordinate (B2) at (6,-2);
            \node[draw,thick,circle,minimum width = 2.5em] (QA2) at (A2) {};
            \node[draw,thick,circle,minimum width = 2.5em] (QB2) at (B2) {};
            \draw[thick,->,>=latex] (QA2) to[out=70,in=110,looseness=6] node[midway, above]{$0|2$} (QA2);
            \draw[thick,->,>=latex] (QA2) to[bend right] node[midway, left]{\begin{tabular}{c} $1|1$ \\ $2|0$ \end{tabular}} (QB2);
            \draw[thick,->,>=latex] (QB2) to[out=250,in=290,looseness=6] node[midway, below]{$0|2$} (QB2);
            \draw[thick,->,>=latex] (QB2) to[bend right] node[midway, right]{\begin{tabular}{c} $1|0$ \\ $2|1$ \end{tabular}} (QA2);

            \draw (8,0) node {$\times$};

            \coordinate (A3) at (10,2);
            \coordinate (B4) at (10,-2);
            \node[draw,thick,circle,minimum width = 2.5em] (QA3) at (A3) {};
            \node[draw,thick,circle,minimum width = 2.5em] (QB4) at (B4) {};
            \draw[thick,->,>=latex] (QA3) to[out=70,in=110,looseness=6] node[midway, above]{$1|0$} (QA3);
            \draw[thick,->,>=latex] (QA3) to[bend right] node[midway, left]{\begin{tabular}{c} $0|2$ \\ $2|1$ \end{tabular}} (QB4);
            \draw[thick,->,>=latex] (QB4) to[out=250,in=290,looseness=6] node[midway, below]{$1|0$} (QB4);
            \draw[thick,->,>=latex] (QB4) to[bend right] node[midway, right]{\begin{tabular}{c} $0|1$ \\ $2|2$ \end{tabular}} (QA3);

            % conjugaison

            \draw[thick,->,>=latex] (8, -4) to node[midway, left] {$\mathfrak{c}$} (8, -5);

            % produit conjugué

            \coordinate (A3) at (6,-7);
            \coordinate (B4) at (6,-11);
            \node[draw,thick,circle,minimum width = 2.5em] (QA3) at (A3) {};
            \node[draw,thick,circle,minimum width = 2.5em] (QB4) at (B4) {};
            \draw[thick,->,>=latex] (QA3) to[out=70,in=110,looseness=6] node[midway, above]{$1|0$} (QA3);
            \draw[thick,->,>=latex] (QA3) to[bend right] node[midway, left]{\begin{tabular}{c} $0|2$ \\ $2|1$ \end{tabular}} (QB4);
            \draw[thick,->,>=latex] (QB4) to[out=250,in=290,looseness=6] node[midway, below]{$1|0$} (QB4);
            \draw[thick,->,>=latex] (QB4) to[bend right] node[midway, right]{\begin{tabular}{c} $0|1$ \\ $2|2$ \end{tabular}} (QA3);

            \draw (8,-9) node {$\times$};

            \coordinate (A2) at (10,-7);
            \coordinate (B2) at (10,-11);
            \node[draw,thick,circle,minimum width = 2.5em] (QA2) at (A2) {};
            \node[draw,thick,circle,minimum width = 2.5em] (QB2) at (B2) {};
            \draw[thick,->,>=latex] (QA2) to[out=70,in=110,looseness=6] node[midway, above]{$0|2$} (QA2);
            \draw[thick,->,>=latex] (QA2) to[bend right] node[midway, left]{\begin{tabular}{c} $1|1$ \\ $2|0$ \end{tabular}} (QB2);
            \draw[thick,->,>=latex] (QB2) to[out=250,in=290,looseness=6] node[midway, below]{$0|2$} (QB2);
            \draw[thick,->,>=latex] (QB2) to[bend right] node[midway, right]{\begin{tabular}{c} $1|0$ \\ $2|1$ \end{tabular}} (QA2);

            %égalité

            \draw (4,-9) node {$=$};

            %conjugué

            \coordinate (A2) at (-2,-7);
            \coordinate (B2) at (-2,-11);
            \coordinate (C2) at (2,-7);
            \coordinate (D2) at (2,-11);
            \node[draw,thick,circle,minimum width = 2.5em] (QA2) at (A2) {$b$};
            \node[draw,thick,circle,minimum width = 2.5em] (QB2) at (B2) {$a$};
            \node[draw,thick,circle,minimum width = 2.5em] (QC2) at (C2) {$c$};
            \node[draw,thick,circle,minimum width = 2.5em] (QD2) at (D2) {$d$};
            \draw[thick,->,>=latex] (QA2) to[out=115,in=155,looseness=6] node[midway, above left]{$1|2$} (QA2);
            \draw[thick,->,>=latex] (QA2) to[bend left] node[midway, above]{\begin{tabular}{c} $0|1$ \\ $2|0$ \end{tabular}} (QC2);
            \draw[thick,->,>=latex] (QB2) to[out=205,in=245,looseness=6] node[midway, below left]{$1|2$} (QB2);
            \draw[thick,->,>=latex] (QB2) to[bend left] node[midway, above]{\begin{tabular}{c} $0|0$ \\ $2|1$ \end{tabular}} (QD2);
            \draw[thick,->,>=latex] (QC2) to[out=25,in=65,looseness=6] node[midway, above right]{$1|2$} (QC2);
            \draw[thick,->,>=latex] (QC2) to[bend left] node[midway, below]{\begin{tabular}{c} $0|1$ \\ $2|0$ \end{tabular}} (QA2);
            \draw[thick,->,>=latex] (QD2) to[out=295,in=335,looseness=6] node[midway, below right]{$1|2$} (QD2);
            \draw[thick,->,>=latex] (QD2) to[bend left] node[midway, below]{\begin{tabular}{c} $0|0$ \\ $2|1$ \end{tabular}} (QB2);
            \end{tikzpicture}
        }
    \end{center}
  \end{figure}
\end{frame}

\begin{frame}
  \frametitle{Une proposition d'extention de la $\mathfrak{md}$-réduction}

  \begin{defi}
    Deux automates $\mathcal{A}$ et $\mathcal{B}$ sont \textbf{\textit{conjugués}} s'il existe des automates $\mathcal{C}$ et $\mathcal{D}$ tels que $\mathcal{A}=\mathcal{CD}$ et $\mathcal{B}=\mathcal{DC}$.
  \end{defi}

  \begin{defi}
    Une paire d'automates duaux est $\mathfrak{mdc}$-réduite si elle est $\mathfrak{md}$-réduite et que les deux automates sont irréductibles.
    La $\mathfrak{mdc}$-réduction consiste à répéter la $\mathfrak{md}$-réduction de l'automate et de ses conjugués jusqu'à trouver des paires $\mathfrak{mdc}$-réduites.
  \end{defi}

\end{frame}

\begin{frame}
  \frametitle{Objectif du projet~ii}

  Nous voulions donc prouver ou contredire la conjecture suivante.

  \begin{conj}
    Un automate de Mealy biréversible engendre un groupe fini si et seulement sa $\mathfrak{mdc}$-réduction produit (au moins) un $\mathfrak{mdc}$-trivial.
  \end{conj}

\end{frame}

\section{Bien-fondé de la $\mathfrak{mdc}$-réduction}

\begin{frame}
  \frametitle{Conservation du caractère fini par minimisation et dualisation}
  \begin{prop}
    Soit $\mathcal{A}$ un automate de Mealy, alors on a :
    \[ \left<\mathfrak{m}\mathcal{A}\right> = \left<\mathcal{A}\right>. \]
  \end{prop}

  \begin{prop}[Akhavi \emph{et al}~\cite{DBLP:journals/corr/abs-1105-4725}]
    Soit $\mathcal{A}$ un automate de Mealy. Le (semi-)groupe $\left<\mathcal{A}\right>_+$ est fini si et seulement si $\left<\mathfrak{d}\mathcal{A}\right>_+$ est fini.
  \end{prop}
\end{frame}

\begin{frame}
  \frametitle{Conservation du caractère fini par conjugaison}

  \begin{prop} %\label{prop:finitude-c}
    Soient $\mathcal{A}, \mathcal{B}$ des automates de Mealy.
    L'automate $\mathcal{A}\mathcal{B}$ engendre un groupe fini si et seulement si $\mathcal{B}\mathcal{A}$ engendre un groupe fini.
  \end{prop}

  \textbf{Démonstration}
  On suppose $\left<\mathcal{A}\mathcal{B}\right>$ fini.

  Tout élément de $\left<\mathcal{B}\mathcal{A}\right>$ est de la forme
  \[
    \rho_{p_1q_1}\circ\rho_{p_2q_2}\circ\cdots\circ\rho_{p_nq_n}
  \]
  où les $p_i$ sont des éléments de $\mathcal{Q}_\mathcal{B}$ et les $q_i$ des éléments de $\mathcal{Q}_\mathcal{A}$.
\end{frame}

\begin{frame}
  \begin{align*}
    \rho_{p_1q_1}\circ\rho_{p_2q_2}\circ\cdots\circ\rho_{p_nq_n} &= \rho_{q_1}\circ\rho_{p_1}\circ\rho_{q_2}\circ\rho_{p_2}\circ\cdots\circ\rho_{q_n}\circ\rho_{p_n} \\
    &=\rho_{q_1}\circ\left(\rho_{p_1}\circ\rho_{q_2}\circ\rho_{p_2}\circ\cdots\circ\rho_{q_n}\right)\circ\rho_{p_n} \\
    &=\rho_{q_1}\circ\underbrace{\left(\rho_{q_2p_1}\circ\rho_{q_3p_2}\circ\cdots\circ\rho_{q_np_{n-1}}\right)}_{\in\left<\mathcal{A}\mathcal{B}\right>}\circ\rho_{p_n}
  \end{align*}

  On en déduit que tout élément de $\left<\mathcal{B}\mathcal{A}\right>$ s'écrit sous la forme
  \[ \rho_q\circ\rho\circ\rho_p. \] Or il y a un nombre fini de $p\in\mathcal{Q}_\mathcal{B}$, de $q\in\mathcal{Q}_\mathcal{A}$ et par hypothèse de $\rho\in\left<\mathcal{AB}\right>$. On en conclut qu'il y en a un nombre fini.

\end{frame}

\begin{frame}
  \frametitle{Un premier résultat sur la $\mathfrak{mdc}$-réduction}
  \begin{prop}
    Toute machine de Mealy engendre un (semi-)groupe fini si et seulement si son $\mathfrak{mdc}$-réduit engendre un (semi-)groupe fini.
  \end{prop}

  \begin{prop}
    Toute machine de Mealy $\mathfrak{mdc}$-trivial engendre un (semi-)groupe fini.
  \end{prop}

  Nous avons donc prouvé une des implications de la conjecture~\ref{conj:birev-mdc}.
\end{frame}

\section{Algorithme de Factorisation}

\begin{frame}
  \frametitle{Clôture par facteurs}
  \begin{prop}[Clôture des inversibles par facteur]
    Soit $\mathcal{A}$ un automate de Mealy \textbf{inversible} qui se factorise en deux automates $\mathcal{A}_1$ et $\mathcal{A}_2$. Alors ces deux automates sont aussi inversibles.
  \end{prop}

  \begin{proof}
    Pour chaque état $(p, r)$ de $\mathcal{A}$, $\rho_{(p, r)}\in \mathfrak{S}(\Sigma)$. Comme $\mathcal{A}=\mathcal{A}_1\cdot\mathcal{A}_2$, on a l'égalité $\rho_{(p, r)}=\rho_{2,r}\circ\rho_{1,p}$.

    Or $\rho_{(p, r)}$ est une bijection donc $\rho_{2,r}$ est surjective et ${\rho_{1,p}}$ est injective. Or puisque ce sont des fonctions de $\Sigma$ dans $\Sigma$ qui est fini on a l'équivalence
    \[ injective \iff surjective \iff bijective. \]
    D'où le résultat.
  \end{proof}

\end{frame}

\begin{frame}
  \frametitle{Clôture par facteurs}
  \begin{prop}[Clôture des réversibles par facteur]
    Soit $\mathcal{A}$ un automate de Mealy \textbf{réversible} qui se factorise en deux automates $\mathcal{A}_1$ et $\mathcal{A}_2$. Alors
    \begin{itemize}
    \item $\mathcal{A}_1$ est réversible.
    \item si $\mathcal{A}_1$ est inversible, alors $\mathcal{A}_2$ est réversible.
    \end{itemize}
  \end{prop}

  \begin{prop}[Clôture des biréversibles par facteur]
    Soit $\mathcal{A}$ un automate \textbf{biréversible} et $\mathcal{A}_1$, $\mathcal{A}_2$ des automates tels que $\mathcal{A}=\mathcal{A}_1\cdot\mathcal{A}_2$, alors $\mathcal{A}_1$ et $\mathcal{A}_2$ sont biréversibles.
  \end{prop}

\end{frame}

\begin{frame}
  \frametitle{Description de l'algorithme}
  \begin{figure}[h!]
    \begin{subfigure}[b]{0.4\textwidth}
      \centering
      \begin{tikzpicture}
        \draw [->,>=latex] (-0.5, 0) node[left] {$p$} to (0.5, 0) node[right] {$q$};
        \draw [->,>=latex] (0, 0.5) node[above] {$x$} to (0, -0.5) node[below] {$y$};
      \end{tikzpicture}
      \caption{Une flèche de l'automate $\mathcal{M}$}
    \end{subfigure}
    ~
    \begin{subfigure}[b]{0.4\textwidth}
      \centering
      \begin{tikzpicture}
        \draw [->,>=latex] (-0.5, 1) node[left] {$p_0$} to (0.5, 1) node[right] {$q_0$};
        \draw [->,>=latex] (0, 1.5) node[above] {$x$} to (0, 0.5);

        \draw [->,>=latex] (-0.5, -0.5) node[left] {$p_1$} to (0.5, -0.5) node[right] {$q_1$};
        \draw [->,>=latex] (0, 0) node[above] {$?$} to (0, -1) node[below] {$y$};
      \end{tikzpicture}
      \caption{En haut l'automate $\mathcal{A}$ et en bas $\mathcal{B}$\label{fig:factor-ab}}
    \end{subfigure}
    \caption{Correspondance entre les flèches dans $\mathcal{M}$ et celles de $\mathcal{A}$ et $\mathcal{B}$. Ici on a que $\iota(p) = (p_0,~p_1)$ et $\iota(q)=(q_0, q_1)$\label{fig:facto}.}
  \end{figure}

\end{frame}

\section{Algorithme de Génération}

\begin{frame}
  \frametitle{Graphe en hélice}
  \begin{figure}[!ht]
    \begin{center}
      \scalebox{1}{
        \begin{tikzpicture}[scale=0.63]
          \coordinate (A) at (-10, 0);
          \coordinate (B) at (-7, 0);

          \node[draw,circle,minimum width = 2.5em] (q_0) at (A) {$a$};
          \node[draw,circle,minimum width = 2.5em] (q_1) at (B) {$b$};
          \path[->,>=latex]
          (q_0) edge [loop left] node {$0|1$} (q_0)
          (q_0) edge [bend left] node[midway, above] {$1|1$} (q_1)
          (q_1) edge [bend left] node[midway, below] {$0|0$} (q_0)
          (q_1) edge [loop right] node {$1|0$} (q_1);

          % ############################################

          \coordinate (A) at (-1,-1);
          \coordinate (B) at (1,-1);
          \coordinate (C) at (1,1);
          \coordinate (D) at (-1,1);

          \coordinate (P1) at (-3,0);
          \coordinate (P2) at (3,0);
          \coordinate (PA) at (0,-3);
          \coordinate (PB) at (0,3);

          \node[draw,circle,minimum width = 2.5em] (QA) at (A) {$a|0$};
          \node[draw,circle,minimum width = 2.5em] (QB) at (B) {$a|1$};
          \node[draw,circle,minimum width = 2.5em] (QC) at (C) {$b|1$};
          \node[draw,circle,minimum width = 2.5em] (QD) at (D) {$b|0$};

          \draw[->,>=latex] (QA) to[bend right] (QB);
          \draw[->,>=latex] (QB) to[bend right] (QC);
          \draw[->,>=latex] (QC) to[bend right] (QD);
          \draw[->,>=latex] (QD) to[bend right] (QA);
        \end{tikzpicture}
      }
    \end{center}
    \caption{Une machine de Mealy et son graphe en hélice.}
  \end{figure}
\end{frame}

\begin{frame}
  \frametitle{Description de l'algorithme}
  Notre algorithme de génération se base sur le résultat suivant:

  \begin{prop}[Akhavi \emph{et al}~\cite{DBLP:journals/corr/abs-1105-4725}]
    Soit $\mathcal{A}$ un automate de Mealy inversible réversible. Les propositions suivantes sont équivalentes~:

    \begin{enumerate}[(i)]
    \item $\mathcal{A}$ est biréversible
    \item Le graphe en hélice de $\mathcal{A}$ est une union de cycles disjoints.
    \end{enumerate}
  \end{prop}

\end{frame}

\begin{frame}[fragile]
\begin{figure}[!ht]
\begin{center}
\begin{Verbatim}[fontsize=\small]

def rec(start, prev, sources, targets, delta, rho):
    if not sources and not targets:
        return delta, rho
    if not prev:
        start = sources.pop()

    # on backtrack sur les targets
    for _ in range(len(targets)):
        p_next, x_next = targets.pop(0)
        delta[p_prev][x_prev] = p_next
        rho[p_prev][x_prev] = x_next

        if valid_delta(delta) and valid_rho(rho):
          res = rec(...)
\end{Verbatim}
\end{center}
  \caption{Structure globale de la fonction de génération en python\label{fig:gen-pseudo-code}}
\end{figure}
\end{frame}

\begin{frame}
  \frametitle{Benchmarks et résultats}
  \begin{table}[h!]
  \begin{center}
    \begin{threeparttable}
      \begin{tabular}{|rrrr|}
        \hline
        \#états & \#lettres & Temps d'exécution & \#biréversibles \\ [0.5ex]
        \hline\hline
        2 & 2 & 0.002s & 12 \\
        \hline
        3 & 2 & 0.003s & 144 \\
        \hline
        3 & 3 & 0.011s & 8 784 \\
        \hline
        4 & 3 & 0.156s & 1 092 096 \\
        \hline
        5 & 3 & 21s    & 16 128 000 \\
        \hline
        4 & 4 & 1m51s  & 1 031 000 000 \\
        \hline
        6 & 3 & 145m44s& 9 848 143 872 \\
        \hline
      \end{tabular}

      \caption{Benchmark génération d'automates biréversibles}
    \end{threeparttable}
  \end{center}
\end{table}
\end{frame}

\begin{frame}
  On atteint des temps d'exécutions très raisonnables.

  On constate que le nombre d'automates biréversible est très grand. Pour pouvoir tous les analyser, il serait préférable de se contenter de les regarder à isomorphime près.

  Pour ce faire, nous avons utilisé la bibliothèque de graphe Nauty~\cite{Nauty}.
\end{frame}

\begin{frame}
  \frametitle{Graphe en hélice augmenté}

  \begin{figure}[!ht]
    \begin{center}
      \scalebox{1}{
        \begin{tikzpicture}[scale=0.63]
          \coordinate (A) at (-10, 0);
          \coordinate (B) at (-7, 0);

          \node[draw,circle,minimum width = 2.5em] (q_0) at (A) {$a$};
          \node[draw,circle,minimum width = 2.5em] (q_1) at (B) {$b$};
          \path[->,>=latex]
          (q_0) edge [loop left] node {$0|1$} (q_0)
          (q_0) edge [bend left] node[midway, above] {$1|1$} (q_1)
          (q_1) edge [bend left] node[midway, below] {$0|0$} (q_0)
          (q_1) edge [loop right] node {$1|0$} (q_1);

          % ############################################

          \coordinate (A) at (-1,-1);
          \coordinate (B) at (1,-1);
          \coordinate (C) at (1,1);
          \coordinate (D) at (-1,1);

          \coordinate (P1) at (-3,0);
          \coordinate (P2) at (3,0);
          \coordinate (PA) at (0,-3);
          \coordinate (PB) at (0,3);

          \node[draw,circle,minimum width = 2.5em] (QA) at (A) {$a|0$};
          \node[draw,circle,minimum width = 2.5em] (QB) at (B) {$a|1$};
          \node[draw,circle,minimum width = 2.5em] (QC) at (C) {$b|1$};
          \node[draw,circle,minimum width = 2.5em] (QD) at (D) {$b|0$};

          \draw[->,>=latex] (QA) to[bend right] (QB);
          \draw[->,>=latex] (QB) to[bend right] (QC);
          \draw[->,>=latex] (QC) to[bend right] (QD);
          \draw[->,>=latex] (QD) to[bend right] (QA);

          \pause

          \node[draw,circle,minimum width = 2.5em,fill=black!20!green] (F1) at (P1) {$0$};
          \node[draw,circle,minimum width = 2.5em,fill=black!20!green] (F2) at (P2) {$1$};


          \draw[->,>=latex] (F1) to[bend right] (QA);
          \draw[->,>=latex] (F1) to[bend left] (QD);
          \draw[->,>=latex] (F2) to[bend left] (QB);
          \draw[->,>=latex] (F2) to[bend right] (QC);

          \pause

          \node[draw,circle,minimum width = 2.5em,fill=white!20!orange] (FA) at (PA) {$a$};
          \node[draw,circle,minimum width = 2.5em,fill=white!20!orange] (FB) at (PB) {$b$};

          \draw[->,>=latex] (FA) to[bend right] (QB);
          \draw[->,>=latex] (FA) to[bend left] (QA);
          \draw[->,>=latex] (FB) to[bend left] (QC);
          \draw[->,>=latex] (FB) to[bend right] (QD);
        \end{tikzpicture}
      }
    \end{center}
    % \caption{Une machine de Mealy et son graphe en hélice augmenté\label{fig:helix-aug}}
  \end{figure}
\end{frame}

\begin{frame}
    \frametitle{Benchmarks et résultats à isomorphisme près}

\begin{table}[h!]
  \begin{center}
    \begin{threeparttable}
      \begin{tabular}{|rrrr|}
        \hline
        \#états & \#lettres & Temps d'exécution & \#birev. à iso. près\\ [0.5ex]
        \hline\hline
        2 & 2 & 0.002s & 8 \\
        \hline
        3 & 2 & 0.003s & 28 \\
        \hline
        3 & 3 & 0.011s & 335  \\
        \hline
        4 & 3 & 2.59s & 8 606 \\
        \hline
        5 & 3 & 12m34s    & 347 753 \\
        \hline
        4 & 4 & 56min34s  & 1 831 489 \\
        \hline
        6 & 3 & $\infty$  &  \\
        \hline
      \end{tabular}

      \caption{Benchmark génération d'automates biréversibles à isomorphisme près\label{table:birev-iso}}
    \end{threeparttable}
  \end{center}
\end{table}

\end{frame}

\section{Avancée sur le problème de finitude}

\begin{frame}
  \frametitle{Fonction de croissance}

  \begin{defi} %\label{def:mass}
    Soit $\mathcal{A}$ une machine de Mealy. On appelle \textbf{\textit{sa fonction de croissance}} la fonction $\pi:\mathbb{N}\rightarrow\mathbb{N}$ qui à un entier $n$ associe le nombre d'états de~$\mathfrak{m}\left(\mathcal{A}^n\right)$.
  \end{defi}

  \begin{prop} %\label{prop:mass}
    Si la fonction de croissance d'une machine de Mealy $\mathcal{A}$ est bornée, alors $\left<\mathcal{A}\right>_+$ est fini.
  \end{prop}
\end{frame}

\begin{frame}
  \frametitle{Vérification empirique de l'hypothèse}

  \begin{enumerate}
  \item engendrer une classe d'automates biréversibles irréductibles à isomorphismes près;
  \item calculer la borne supérieure des fonctions de croissance des automates $\mathfrak{md}$-triviaux;
  \item vérifier que la fonction de croissance de tous les automates non $\mathfrak{md}$-trivaux dépasse cette borne.
  \end{enumerate}
\end{frame}

\begin{frame}
  \frametitle{Contre-exemple}
  \begin{center}
    \begin{figure}[!ht]
      \begin{center}
        \scalebox{0.8}{
          \begin{tikzpicture}
            \coordinate (A) at (270:4);
            \coordinate (B) at (0,0);
            \coordinate (C) at (180:4);
            \coordinate (D) at (0:4);
            \coordinate (E) at (90:4);

            \node[draw,thick,circle,minimum width = 2.5em] (QA) at (A) {$a$};
            \node[draw,thick,circle,minimum width = 2.5em] (QB) at (B) {$b$};
            \node[draw,thick,circle,minimum width = 2.5em] (QC) at (C) {$c$};
            \node[draw,thick,circle,minimum width = 2.5em] (QD) at (D) {$d$};
            \node[draw,thick,circle,minimum width = 2.5em] (QE) at (E) {$e$};

            \draw[thick,->,>=latex] (QA) to[bend left] node[midway, left]{$0|1$} (QB);
            \draw[thick,->,>=latex] (QA) to[bend left] node[midway, below left]{$2|0$} (QC);
            \draw[thick,->,>=latex] (QA) to node[midway, above left]{$1|2$} (QD);
            \draw[thick,->,>=latex] (QB) to[bend left] node[midway, left]{$1|0$} (QA);
            \draw[thick,->,>=latex] (QB) to node[midway, above]{$0|2$} (QC);
            \draw[thick,->,>=latex] (QB) to[bend left] node[midway, above]{$2|1$} (QD);
            \draw[thick,->,>=latex] (QC) to[bend left] node[midway,above left]{\begin{tabular}{c} $0|1$ \\ $1|0$ \\ $2|2$ \end{tabular}} (QE);
            \draw[thick,->,>=latex] (QD) to[bend left] node[midway, below right]{$2|1$} (QA);
            \draw[thick,->,>=latex] (QD) to[bend left] node[midway, above]{$1|2$} (QB);
            \draw[thick,->,>=latex] (QD) to[out=340,in=20,looseness=6] node[midway, right]{$0|0$} (QD);
            \draw[thick,->,>=latex] (QE) to[out=0,in=0,looseness=2.3] node[midway, right]{$0|2$} (QA);
            \draw[thick,->,>=latex] (QE) to node[midway, right]{$2|0$} (QB);
            \draw[thick,->,>=latex] (QE) to node[midway, below right]{$1|1$} (QC);
          \end{tikzpicture}
        }
        % \caption{Contre-exemple à la conjecture.\label{fig:contre-exemple}}
      \end{center}
    \end{figure}
  \end{center}
\end{frame}

\section{Conclusion}

\begin{frame}[allowframebreaks]
  \frametitle{Références}
  \bibliography{project}{
    \nocite{*}
  }
  \bibliographystyle{plain}
\end{frame}

\end{document}