\documentclass[11pt]{beamer}
\usepackage[utf8]{inputenc}
\usepackage[T1]{fontenc}
\usepackage{lmodern}
\usepackage[french]{babel}
\usetheme{metropolis}

\usepackage{amsmath}
\usepackage{amssymb}
\usepackage{amsthm}
\usepackage{mathtools}

\newtheorem{prop}{Proposition}
\newtheorem{defi}{Définition}
\newtheorem{thm}{Théorème}
\newtheorem{conj}{Conjecture}
\newtheorem*{rem*}{Remarque}
\begin{document}
	\author{Maxime Flin  \& Tristan François}
	\title{\`A propos du problème de finitude des groupes d'automates biréversibles}
	%\subtitle{}
	%\logo{}
	%\institute{}
	%\date{}
	%\subject{}
	%\setbeamercovered{transparent}
	%\setbeamertemplate{navigation symbols}{}
	\begin{frame}[plain]
		\maketitle
	\end{frame}
	
	\section{Automates de Mealy et position du problème}
	
	\begin{frame}
		\frametitle{Objectifs du Projet}
	\end{frame}

	\begin{frame}
		\frametitle{Automate de Mealy}
	\end{frame}

	\begin{frame}
		\frametitle{Minimisation et Dualisation : $\mathfrak{md}$-réduction}
	\end{frame}

	\begin{frame}
		\frametitle{Limite et nouvelle opération : $\mathfrak{mdc}$-réduction}
	\end{frame}

	\section{Bien fondé de la $\mathfrak{mdc}$-réduction}
	
	\begin{frame}
		\frametitle{$\left<\mathcal{A}\mathcal{B}\right>$ fini $\Leftrightarrow \left<\mathcal{B}\mathcal{A}\right>$ fini}
	\end{frame}

	\section{Algorithme de Factorisation}
	
	\begin{frame}
		\frametitle{Clôture des automates par facteurs}
	\end{frame}

	\begin{frame}
		\frametitle{Description de l'algorithme}
	\end{frame}

	\section{Algorithme de Génération}
	
	\begin{frame}
		\frametitle{Graphe en hélice augmenté}
	\end{frame}

	\begin{frame}
		\frametitle{Description de l'algorithme}
	\end{frame}

	\begin{frame}
		\frametitle{Benchmarks et résultats}
	\end{frame}

	\section{Avancée sur le problème de finitude}
	
	\begin{frame}
		\frametitle{Fonction de croissance}
		
		\begin{defi} \label{def:mass}
			Soit $\mathcal{A}$ une machine de Mealy. On appelle \textbf{\textit{sa fonction de croissance}} la fonction $\pi:\mathbb{N}\rightarrow\mathbb{N}$ qui à un entier $n$ associe le nombre d'états de~$\mathfrak{m}\left(\mathcal{A}^n\right)$.
		\end{defi}
		
		\begin{prop} \label{prop:mass}
			Si la fonction de croissance d'une machine de Mealy $\mathcal{A}$ est bornée, alors $\left<\mathcal{A}\right>_+$ est fini.
		\end{prop}
	\end{frame}

	\begin{frame}
		\frametitle{Vérification empirique de l'hypothèse}
		
		\begin{enumerate}
			\item engendrer une classe d'automates biréversibles irréductibles à isomorphismes près;
			\item calculer la borne supérieure des fonctions de croissance des automates $\mathfrak{md}$-triviaux;
			\item vérifier que la fonction de croissance de tous les automates non $\mathfrak{md}$-trivaux dépasse cette borne.
		\end{enumerate}
	\end{frame}

	\begin{frame}
		\frametitle{Contre-exemple}
		\begin{center}
		\begin{figure}[!ht]
			\begin{center}
				\scalebox{0.8}{
					\begin{tikzpicture}
					\coordinate (A) at (270:4);
					\coordinate (B) at (0,0);
					\coordinate (C) at (180:4);
					\coordinate (D) at (0:4);
					\coordinate (E) at (90:4);
					
					\node[draw,thick,circle,minimum width = 2.5em] (QA) at (A) {$a$};
					\node[draw,thick,circle,minimum width = 2.5em] (QB) at (B) {$b$};
					\node[draw,thick,circle,minimum width = 2.5em] (QC) at (C) {$c$};
					\node[draw,thick,circle,minimum width = 2.5em] (QD) at (D) {$d$};
					\node[draw,thick,circle,minimum width = 2.5em] (QE) at (E) {$e$};
					
					\draw[thick,->,>=latex] (QA) to[bend left] node[midway, left]{$0|1$} (QB);
					\draw[thick,->,>=latex] (QA) to[bend left] node[midway, below left]{$2|0$} (QC);
					\draw[thick,->,>=latex] (QA) to node[midway, above left]{$1|2$} (QD);
					\draw[thick,->,>=latex] (QB) to[bend left] node[midway, left]{$1|0$} (QA);
					\draw[thick,->,>=latex] (QB) to node[midway, above]{$0|2$} (QC);
					\draw[thick,->,>=latex] (QB) to[bend left] node[midway, above]{$2|1$} (QD);
					\draw[thick,->,>=latex] (QC) to[bend left] node[midway,above left]{\begin{tabular}{c} $0|1$ \\ $1|0$ \\ $2|2$ \end{tabular}} (QE);
					\draw[thick,->,>=latex] (QD) to[bend left] node[midway, below right]{$2|1$} (QA);
					\draw[thick,->,>=latex] (QD) to[bend left] node[midway, above]{$1|2$} (QB);
					\draw[thick,->,>=latex] (QD) to[out=340,in=20,looseness=6] node[midway, right]{$0|0$} (QD);
					\draw[thick,->,>=latex] (QE) to[out=0,in=0,looseness=2.3] node[midway, right]{$0|2$} (QA);
					\draw[thick,->,>=latex] (QE) to node[midway, right]{$2|0$} (QB);
					\draw[thick,->,>=latex] (QE) to node[midway, below right]{$1|1$} (QC);
					\end{tikzpicture}
				}
				%\caption{Contre-exemple à la conjecture.\label{fig:contre-exemple}}
			\end{center}
		\end{figure}
		\end{center}
	\end{frame}

	\section{Conclusion}
\end{document}