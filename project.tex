\documentclass{article}

\usepackage[francais]{babel}

\usepackage{amsmath}
\usepackage{amsthm}

\newtheorem{prop}{Proposition}
\newtheorem{thm}{Théorème}

\title{À propos du problème de finitude des (semi)groupes d'automates}
\author{Tristan François \& Maxime Flin}

\begin{document}
\maketitle

\begin{abstract}
  Compte rendu du projet de dernière année de la licence de Mathématique et d'informatique.
\end{abstract}


\section*{Introduction}
Blablabla automate

\section{Automates de Mealy}

\subsection{Cloture de classes d'automates}
\begin{prop}[Cloture des inversibles par produit]
  Soient $\mathcal{A}$ et $\mathcal{B}$ des automates de Mealy \textbf{inversibles}. Alors $\mathcal{A}\cdot\mathcal{B}$ est inversible
\end{prop}

\begin{proof}
  Soient $\mathcal{A}=\left(\mathcal{Q}, \Sigma, \delta, \rho\right)$ et $\mathcal{B}=\left(\mathcal{Q'}, \Sigma, \delta', \rho'\right)$ deux automates inversibles, et soit $\mathcal{A\cdot B}=\left(\mathcal{Q\times Q'}, \Sigma, \delta'', \rho''\right)$ leur produit.


  Considérons un état $(p, r)$ de ce produit. Alors $\rho_p\circ\rho'_r=\rho_{(p,r)}$, or puisque les automates $\mathcal{A}$ et $\mathcal{B}$ sont inversibles, $\rho_p$ et $\rho'_r$ sont des permutations du groupe de symétrie de $\Sigma$, on en déduit que $\rho_{(p, r)}$ en est une aussi.


  On a bien que tous les $(\rho''_q)_{q\in Q\times Q'}$ sont des permutations sur les lettres, c'est à dire que l'automate est inversible.
\end{proof}

\begin{prop}[Cloture des réversibles par produit]
  Soient $\mathcal{A}$ et $\mathcal{B}$ des automates de Mealy \textbf{réversibles}. Si $\mathcal{A}$ est \textbf{inversible}, alors $\mathcal{A}\cdot\mathcal{B}$ est réversible.
\end{prop}

\begin{proof}
  Soient $\mathcal{A}=\left(\mathcal{Q}, \Sigma, \delta, \rho\right)$ et $\mathcal{B}=\left(\mathcal{Q'}, \Sigma, \delta', \rho'\right)$ deux automates réversibles, et soit $\mathcal{A\cdot B}=\left(\mathcal{Q\times Q'}, \Sigma, \delta'', \rho''\right)$ leur produit.


  On suppose $\mathcal{A}$ inversible.


  Par définition $\delta''_x(pr)=(\delta_x(p), \delta'_{\rho_p(x)}(r))$. Or, puisque les automates sont réversibles, les $(\delta_x)_{x\in\Sigma}$ et $(\delta'_x)_{x\in\Sigma}$ sont des bijections. De plus, $\mathcal{A}$ esr inversible donc les $(\rho_p)_{p\in \mathcal{Q}}$ sont des bijections. On en conclut que les $(\delta''_{pr})_{pr\in\mathcal{Q}\times\mathcal{Q'}}$ sont aussi des bijections.
\end{proof}

\begin{prop}[Clôture des biréversibles par produit]
  Soient $\mathcal{A}$ et $\mathcal{B}$ des automates \textbf{biréversibles}, alors $\mathcal{A}\cdot\mathcal{B}$ est \textbf{biréversible}.
\end{prop}

\begin{prop}[Cloture des inversibles par facteurs]
  Soit $\mathcal{A}$ un automate de Mealy \textbf{inversible} qui se factorise en deux automates $\mathcal{A}_1$ et $\mathcal{A}_2$. Alors ces deux automates sont aussi inversibles.
\end{prop}

\begin{proof}
  Soient $\mathcal{A}=\left(\mathcal{Q\cdot Q'}, \Sigma, \delta, \rho\right)$, $\mathcal{A}_1=\left(\mathcal{Q}, \Sigma, \delta', \rho'\right)$ et $\mathcal{A}_2=\left(\mathcal{Q'}, \Sigma, \delta'', \rho''\right)$ tel que ci dessus.

  Pour chaque état $(p, r)$ de $\mathcal{A}$, $\rho_{(p, r)}\in S(\Sigma)$. Comme $\mathcal{A}=\mathcal{A}_1\cdot\mathcal{A}_2$, on a l'égalité $\rho_{(p, r)}=\rho'_p\circ\rho''_r$. Puisque $\rho_{(p, r)}$ est une permutation, on en déduit que $\rho'_p$ et $\rho''_r$ en sont aussi, d'où le résultat.
\end{proof}

\begin{prop}[Cloture des réversibles par facteurs]
  Soit $\mathcal{A}$ un automate de Mealy \textbf{réversible} qui se factorise en deux automates $\mathcal{A}_1$ et $\mathcal{A}_2$. Alors
  \begin{itemize}
  \item $\mathcal{A}_1$ est réversible.
  \item si $\mathcal{A}_1$ est inversible, alors $\mathcal{A}_2$ est réversible.
  \end{itemize}
\end{prop}

\begin{proof}
  Soient $\mathcal{A}=\left(\mathcal{Q\cdot Q'}, \Sigma, \delta, \rho\right)$ réversible, $\mathcal{A}_1=\left(\mathcal{Q}, \Sigma, \delta', \rho'\right)$ et $\mathcal{A}_2=\left(\mathcal{Q'}, \Sigma, \delta'', \rho''\right)$ tels que $\mathcal{A} = \mathcal{A}_1\cdot\mathcal{A}_2$.

  Par définition, on a que $\delta_x(pr) = (\delta'_x(p), \delta''_{\rho'_p(x)}(r))$. Les $(\delta_x)_{x\in\Sigma}$ sont inversibles, alors il est clair les $(\delta'_x)_{x\in\Sigma}$ sont inversibles.

  De plus, chacun des $\delta''_{\rho_p(x)}$ est inversibles, donc si $\mathcal{A}_1$ est inversible, les $(\rho_p)_{p\in\Sigma}$ étant inversibles, alors tous les $(\delta''_x)_{x\in\Sigma}$ sont inversibles. D'où $\mathcal{A}_2$ est réversible.

\end{proof}

\begin{prop}[Clôture des biréversibles par facteurs]
  Soit $\mathcal{A}$ un automate \textbf{biréversible} et $\mathcal{A}_1$, $\mathcal{A}_2$ des automates tels que $\mathcal{A}=\mathcal{A}_1\cdot\mathcal{A}_2$, alors $\mathcal{A}_1$ et $\mathcal{A}_2$ sont biréversible.
\end{prop}

\end{document}