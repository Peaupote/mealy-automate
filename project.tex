\documentclass{article}

\usepackage[francais]{babel}
\usepackage[utf8]{inputenc}
\usepackage{amsmath}
\usepackage{amssymb}
\usepackage{amsthm}
\usepackage{mathtools}
\usepackage{enumerate}
\usepackage[T1]{fontenc}
\usepackage[hidelinks]{hyperref}
\usepackage{natbib}

\newtheorem{prop}{Proposition}
\newtheorem{lemma}{Lemme}
\newtheorem{thm}{Théorème}
\newtheorem{definition}{Définition}
\newtheorem{conj}{Conjecture}

\title{À propos du problème de finitude des (semi-)groupes d'automates}
\author{Maxime Flin \& Tristan François}

\begin{document}
\maketitle

\begin{abstract}
  Les automates de Mealy sont une extension des automates qui écrivent des mots en même temps qu'ils en lisent. En étudiant leur action sur un ensemble de mot, on peut dégager une structure de (semi-)groupe engendré par l'automate.

  Le problème qui nous intéresse ici est de pouvoir décider si un automate de Mealy engendre une structure finie ou infinie. Ce problème, qui est indécidable en général, semble abordable pour des classes d'automates plus réduites. En suivant la voie ouverte par \citeauthor{Klimann13}, nous avons essayé de généraliser son résultat en ajoutant la conjugaison à la md-réduction.
\end{abstract}


\section*{Introduction}
Blablabla automate

\newpage

\section{Automates de Mealy}

\subsection{Automates de Mealy et quelques constructions}

La notion d'automate de Mealy étend la simple notion d'automate en lui ajoutant une sortie en écriture.

\begin{definition}
  Un automate de Mealy est la donné d'un quadruplet $\left(\mathcal{Q}, \Sigma, \delta, \rho\right)$ où $\mathcal{Q}$ est l'ensemble des états de l'automate, $\Sigma$ l'alphabet sur lequel l'automate agit, $\delta$ une famille d'applications de $\mathcal{Q}$ dans $\mathcal{Q}$ indexé par $\Sigma$ qui représente les transitions entre les états et $\rho$ une famille d'applications de $\Sigma$ dans $\Sigma$ indexé par $\mathcal{Q}$ qui représente l'écriture en sortie de l'automate.
\end{definition}

% exemple avec figure

\begin{definition}
  Soit $\mathcal{A}=\left(\mathcal{Q}, \Sigma, \delta, \rho\right)$ un automate de Mealy.
  \begin{enumerate}[i)]
  \item On dit que $\mathcal{A}$ est inversible si $\rho_q$ est une permutation de $\Sigma$ pour tout état $q$.
  \item On dit que $\mathcal{A}$ est réversible si $\delta_x$ est une permuration de $\mathcal{Q}$ pour toute lettre $x$.
  \item On dit que $\mathcal{A}$ est biréversible s'il est inversible, réversible et que son inverse est réversible.
  \end{enumerate}
\end{definition}

\begin{definition}
  Soit $\mathcal{A}=\left(\mathcal{Q}, \Sigma, \delta, \rho\right)$ un automate de Mealy. \textit{L'équivalence de Nérode} $\equiv$ sur $\mathcal{Q}$ est la limite de la suite de relations d'équivalences ($\equiv_k$) de plus en plus fines définies récursivement par

  \begin{align*}
    \forall p, q \in Q,\\
    p \equiv_0 q &\iff \rho_p = \rho_q \\
    \forall k \geq 0, p \equiv_{k+1}q &\iff \left(p\equiv_kq \wedge \forall x \in \Sigma,~\delta_x(p)=\delta_x(q)\right)
  \end{align*}
\end{definition}


\subsection{Action sur les mots et (semi)groupes engendrés}

\begin{definition}
  Soit $\mathcal{A}=\left(\mathcal{Q}, \Sigma, \delta, \rho\right)$ un automate de Mealy. On peut étendre l'ensemble de définition de $\rho$ à sur $\Sigma^*$ par induction comme suit.

  Soient $p\in\mathcal{Q}$, $x\in\Sigma$ et $\textbf{u}\in\Sigma^*$
  \begin{align*}
    &\rho_p(\epsilon)=\epsilon \\
    &\rho_p(x\textbf{u})=\rho_p(x)\rho_{\delta_x(p)}(\textbf{u})
  \end{align*}

  On définit le semigroupe d'automate engendré par $\mathcal{A}$ comme
  \begin{equation}
    \left<\mathcal{A}\right>_+=\left<\rho_p, \forall p\in\mathcal{Q}\right>
  \end{equation}

  Si l'automate $\mathcal{A}$ est inversible, alors on peut aussi considérer l'inverse des $\rho_p$. Dans ce cas, l'automate engendre un groupe noté $\left<\mathcal{A}\right>$.

\end{definition}

\begin{prop}
  \label{prop:finitude-d}
  Soit $\mathcal{A}$ un automate de Mealy. $\left<\mathcal{A}\right>$ est fini si et seulement si $\left<d\mathcal{A}\right>$ est fini.
\end{prop}

\begin{proof}
  TODO
\end{proof}

\begin{prop}
  \label{prop:finitude-m}
  Soit $\mathcal{A}$ un automate de Mealy. Si $m(A)$ engendre un groupe fini, alors $\mathcal{A}$ engendre un groupe fini.
\end{prop}

\begin{proof}
  TODO
\end{proof}

Des proposition \ref{prop:finitude-d} et \ref{prop:finitude-m} on déduit la proposition suivante.

\begin{prop}
  \label{prop:md-trivial}
  Tout automate de Mealy dont le md-réduit engendre un groupe fini engendre un groupe fini.
\end{prop}

\begin{conj}
  Un automate de Mealy engendre un groupe fini si et seulement si son md-réduit engendre un groupe fini.
\end{conj}

Cette conjecture est fausse en général. Il semble pourtant que dans la classe particulière des automates biréversibles, la md-réduction est efficace.

\begin{conj}
  \label{conj:birev-md}
  Un automate de Mealy biréversible engendre un groupe fini si et seulement si son md-réduit engendre un groupe fini.
\end{conj}

\begin{thm}(\citeauthor{Klimann13})
  Tout automate fini à deux lettres et/ou deux états engendre un groupe fini si et seulement s'il est md-trivial.
\end{thm}

Malheureusement, la conjecture \ref{conj:birev-md} n'est pas vraie en général.

% TODO: include fantastique figures

Ces contres exemples sont pourtant factorisables, et en conjugant leur factorisation, on trouve un automate md-trivial. Ce qui nous conduit à la conjecture suivante.

\begin{conj}
  \label{conj:birev-mdc}
  Un automate de Mealy biréversible engendre un groupe fini si et seulement si son mdc-réduit engendre un groupe fini.
\end{conj}

Cette conjecture est l'objet des recherches ménées au cours de ce projet.

\begin{prop}
  \label{prop:finitude-c}
  Soient $\mathcal{A}, \mathcal{B}$ des automates de Mealy.
  $\mathcal{A}\mathcal{B}$ engendre un groupe fini si et seulement si $\mathcal{B}\mathcal{A}$ engendre un groupe fini.
\end{prop}

\begin{proof}
  Soient $\mathcal{A}=\left(\mathcal{Q}_1, \Sigma, \delta, \rho\right)$ et $\mathcal{B}=\left(\mathcal{Q}_2, \Sigma, \delta, \rho\right)$ des automates de Mealy sur un même alphabet.

  On suppose sans perte de généralité $\mathcal{Q}_1$ et $\mathcal{Q}_2$ disjoints. On écrit donc $\rho$ pour les deux automates sans ambiguïté.

  On suppose $\left<\mathcal{A}\mathcal{B}\right>$ fini.

  Tout élément de $\left<\mathcal{B}\mathcal{A}\right>$ est de la forme
  \[
    \rho_{p_1q_1}\circ\rho_{p_2q_2}\circ\cdots\circ\rho_{p_nq_n}
  \]
  où les $p_i$ sont des éléments de $\mathcal{Q}_2$ et les $q_i$ des éléments de $\mathcal{Q}_1$. Or
  \begin{align*}
    \rho_{p_1q_1}\circ\rho_{p_2q_2}\circ\cdots\circ\rho_{p_nq_n} &= \rho_{q_1}\circ\rho_{p_1}\circ\rho_{q_2}\circ\rho_{p_2}\circ\cdots\circ\rho_{q_n}\circ\rho_{p_n} \\
    &=\rho_{q_1}\circ\left(\rho_{p_1}\circ\rho_{q_2}\circ\rho_{p_2}\circ\cdots\circ\rho_{q_n}\right)\circ\rho_{p_n} \\
    &=\rho_{q_1}\circ\underbrace{\left(\rho_{q_2p_1}\circ\rho_{q_3p_2}\circ\cdots\circ\rho_{q_np_{n-1}}\right)}_{\in\left<\mathcal{A}\mathcal{B}\right>}\circ\rho_{p_n}
  \end{align*}

  On en déduit que tout élément de $\left<\mathcal{B}\mathcal{A}\right>$ s'écrit comme la composition d'un $\rho_{q},~q\in\mathcal{Q}_1$, d'un $\rho_\bullet\in\left<\mathcal{A}\mathcal{B}\right>$ et d'un $\rho_{p},~p\in\mathcal{Q}_2$. Or $\mathcal{Q}_1$ et $\mathcal{Q}_2$ sont finis, donc le nombre de choix de $p$ et de $q$ sont fini. De plus, on a supposé le nombre d'éléments $\rho_\bullet\in\left<\mathcal{A}\mathcal{B}\right>$ est fini. Donc le nombre d'élements qui s'écrivent \[ \rho_q\circ\rho_\bullet\circ\rho_p \] est fini. On a montré plus haut que tout éléments de $\left<\mathcal{B}\mathcal{A}\right>$ sont de cette forme, on en conclut qu'il y en a un nombre fini.

  $\mathcal{A}$ et $\mathcal{B}$ jouant des rôles complètement symétriques, la même démonstration marche pour montrer la réciproque.
\end{proof}

On déduit des propositions \ref{prop:finitude-d}, \ref{prop:finitude-m} et \ref{prop:finitude-c} une des implications de la conjecture \ref{conj:birev-mdc}.

\begin{prop}
  Tout automate de Mealy dont le mdc-réduit engendre un groupe fini engendre un groupe fini.
\end{prop}

\section{Génération et Factorisation d'automates de Mealy}
La génération et la factorisation d'automates de Mealy est un problème pour lequel aucune solution efficace n'existe pour l'instant. Dans le but de pouvoir infirmer ou consolider la conjecture \ref{conj:birev-mdc}, une partie de notre travail a été d'essayer de trouver et d'implémenter des méthodes de génération et de factorisation efficaces pour les automates de Mealy biréversibles.


\subsection{Clôture de classes d'automates}

Puisque nous nous interéssont à la classe des automates biréversibles, montrons qu'elle est bien close par les opérations qui nous interésse, ie. la dualisation, la minimisation, le produit et la factorisation.

\begin{prop}[Clôture des inversibles par produit]
  Soient $\mathcal{A}$ et $\mathcal{B}$ des automates de Mealy \textbf{inversibles}. Alors $\mathcal{A}\cdot\mathcal{B}$ est inversible
\end{prop}

\begin{proof}
  Soient $\mathcal{A}=\left(\mathcal{Q}, \Sigma, \delta, \rho\right)$ et $\mathcal{B}=\left(\mathcal{Q'}, \Sigma, \delta', \rho'\right)$ deux automates inversibles, et soit $\mathcal{A\cdot B}=\left(\mathcal{Q\times Q'}, \Sigma, \delta'', \rho''\right)$ leur produit.


  Considérons un état $(p, r)$ de ce produit. Alors $\rho_r\circ\rho'_p=\rho_{(p,r)}$, or puisque les automates $\mathcal{A}$ et $\mathcal{B}$ sont inversibles, $\rho_p$ et $\rho'_r$ sont des permutations du groupe de symétrie de $\Sigma$, on en déduit que $\rho_{(p, r)}$ en est une aussi.


  On a bien que tous les ${(\rho''_q)}_{q\in Q\times Q'}$ sont des permutations sur les lettres, c'est à dire que l'automate est inversible.
\end{proof}

\begin{prop}[Clôture des réversibles par produit]
  Soient $\mathcal{A}$ et $\mathcal{B}$ des automates de Mealy \textbf{réversibles}. Si $\mathcal{A}$ est \textbf{inversible}, alors $\mathcal{A}\cdot\mathcal{B}$ est réversible.
\end{prop}

\begin{proof}
  Soient $\mathcal{A}=\left(\mathcal{Q}, \Sigma, \delta, \rho\right)$ et $\mathcal{B}=\left(\mathcal{Q'}, \Sigma, \delta', \rho'\right)$ deux automates réversibles, et soit $\mathcal{A\cdot B}=\left(\mathcal{Q\times Q'}, \Sigma, \delta'', \rho''\right)$ leur produit.


  On suppose $\mathcal{A}$ inversible.


  Par définition $\delta''_x(pr)=(\delta_x(p), \delta'_{\rho_p(x)}(r))$. Or, puisque les automates sont réversibles, les $(\delta_x)_{x\in\Sigma}$ et ${(\delta'_x)}_{x\in\Sigma}$ sont des bijections. De plus, $\mathcal{A}$ esr inversible donc les ${(\rho_p)}_{p\in \mathcal{Q}}$ sont des bijections. On en conclut que les ${(\delta''_{pr})}_{pr\in\mathcal{Q}\times\mathcal{Q'}}$ sont aussi des bijections.
\end{proof}

\begin{prop}[Clôture des biréversibles par produit]
  Soient $\mathcal{A}$ et $\mathcal{B}$ des automates \textbf{biréversibles}, alors $\mathcal{A}\cdot\mathcal{B}$ est \textbf{biréversible}.
\end{prop}

\begin{prop}[Clôture des inversibles par facteurs]\label{prop_cloture_inv_facteurs}
  Soit $\mathcal{A}$ un automate de Mealy \textbf{inversible} qui se factorise en deux automates $\mathcal{A}_1$ et $\mathcal{A}_2$. Alors ces deux automates sont aussi inversibles.
\end{prop}

\begin{proof}
  Soient $\mathcal{A}=\left(\mathcal{Q\cdot Q'}, \Sigma, \delta, \rho\right)$, $\mathcal{A}_1=\left(\mathcal{Q}, \Sigma, \delta', \rho'\right)$ et $\mathcal{A}_2=\left(\mathcal{Q'}, \Sigma, \delta'', \rho''\right)$ tel que ci dessus.

  Pour chaque état $(p, r)$ de $\mathcal{A}$, $\rho_{(p, r)}\in S(\Sigma)$. Comme $\mathcal{A}=\mathcal{A}_1\cdot\mathcal{A}_2$, on a l'égalité $\rho_{(p, r)}=\rho'_r\circ\rho''_p$. Puisque $\rho_{(p, r)}$ est une permutation, on en déduit que $\rho'_p$ et $\rho''_r$ en sont aussi, d'où le résultat.
\end{proof}

\begin{prop}\label{prop_inverse_produit}
    Soit $\mathcal{A}$ un automate de Mealy \textbf{inversible} qui se factorise en deux automates $\mathcal{A}_1$ et $\mathcal{A}_2$.
    Alors $\mathcal{A}^{-1} = \mathcal{A}_2^{-1} \cdot \mathcal{A}_1^{-1}$.
\end{prop}

\begin{proof}
  TODO
\end{proof}

\begin{prop}[Clôture des réversibles par facteurs]\label{prop_cloture_rev_facteurs}
  Soit $\mathcal{A}$ un automate de Mealy \textbf{réversible} qui se factorise en deux automates $\mathcal{A}_1$ et $\mathcal{A}_2$. Alors
  \begin{itemize}
  \item $\mathcal{A}_1$ est réversible.
  \item si $\mathcal{A}_1$ est inversible, alors $\mathcal{A}_2$ est réversible.
  \end{itemize}
\end{prop}

\begin{proof}
  Soient $\mathcal{A}=\left(\mathcal{Q\cdot Q'}, \Sigma, \delta, \rho\right)$ réversible, $\mathcal{A}_1=\left(\mathcal{Q}, \Sigma, \delta', \rho'\right)$ et $\mathcal{A}_2=\left(\mathcal{Q'}, \Sigma, \delta'', \rho''\right)$ tels que $\mathcal{A} = \mathcal{A}_1\cdot\mathcal{A}_2$.

  Par définition, on a que $\delta_x(pr) = (\delta'_x(p), \delta''_{\rho'_p(x)}(r))$. Les ${(\delta_x)}_{x\in\Sigma}$ sont inversibles, alors il est clair les ${(\delta'_x)}_{x\in\Sigma}$ sont inversibles.

  De plus, chacun des $\delta''_{\rho_p(x)}$ est inversibles, donc si $\mathcal{A}_1$ est inversible, les ${(\rho_p)}_{p\in\Sigma}$ étant inversibles, alors tous les ${(\delta''_x)}_{x\in\Sigma}$ sont inversibles. D'où $\mathcal{A}_2$ est réversible.
\end{proof}

\begin{prop}[Clôture des biréversibles par facteurs]
  Soit $\mathcal{A}$ un automate \textbf{biréversible} et $\mathcal{A}_1$, $\mathcal{A}_2$ des automates tels que $\mathcal{A}=\mathcal{A}_1\cdot\mathcal{A}_2$, alors $\mathcal{A}_1$ et $\mathcal{A}_2$ sont biréversible.
\end{prop}

\begin{proof}
    $\mathcal{A}$ est biréversible, donc en particulier $\mathcal{A}$ est inversible et réversible. D'après la proposition \ref{prop_cloture_inv_facteurs}, $\mathcal{A}_1$ et $\mathcal{A}_2$ sont inversibles. Puisque $\mathcal{A}_1$ est inversible, d'après la proposition \ref{prop_cloture_rev_facteurs}, $\mathcal{A}_1$ et $\mathcal{A}_2$ sont réversibles.

    D'après la proposition \ref{prop_inverse_produit}, $\mathcal{A}^{-1} = \mathcal{A}_2^{-1} \cdot \mathcal{A}_1^{-1}$. Or $\mathcal{A}^{-1}$ est réversible puisque $\mathcal{A}$ est biréversible. Donc, toujours d'après la proposition \ref{prop_cloture_rev_facteurs} et puisque $\mathcal{A}_1^{-1}$ est inversible, $\mathcal{A}_1^{-1}$ et $\mathcal{A}_2^{-1}$ sont réversibles.

    Ainsi, $\mathcal{A}_1$ et $\mathcal{A}_2$ sont bien biréversibles.
\end{proof}


\newpage
\bibliography{project}{
  \nocite{*}
}
\bibliographystyle{plain}

\end{document}
