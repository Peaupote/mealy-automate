\documentclass{article}

\usepackage[francais]{babel}

\usepackage{amsmath}
\usepackage{amsthm}

\newtheorem{prop}{Proposition}
\newtheorem{thm}{Théorème}

\title{À propos du problème de finitude des (semi)groupes d'automates}
\author{Tristan François \& Maxime Flin}

\begin{document}
\maketitle

\begin{abstract}
  Compte rendu du projet de dernière année de la licence de Mathématique et d'informatique.
\end{abstract}


\section*{Introduction}
Blablabla automate

\section{Automates de Mealy}

\subsection{Cloture de classes d'automates}
\begin{prop}[Cloture des inversibles par produit]
  Soient $\mathcal{A}$ et $\mathcal{B}$ des automates de Mealy \textbf{inversibles}. Alors $\mathcal{A}\cdot\mathcal{B}$ est inversible
\end{prop}

\begin{proof}
  Soient $\mathcal{A}=\left(\mathcal{Q}, \Sigma, \delta, \rho\right)$ et $\mathcal{B}=\left(\mathcal{Q'}, \Sigma, \delta', \rho'\right)$ deux automates inversibles, et soit $\mathcal{A\cdot B}=\left(\mathcal{Q\times Q'}, \Sigma, \delta'', \rho''\right)$ leur produit.


  Considérons un état $(p, r)$ de ce produit. Alors $\rho_p\circ\rho'_r=\rho_{(p,r)}$, or puisque les automates $\mathcal{A}$ et $\mathcal{B}$ sont inversibles, $\rho_p$ et $\rho'_r$ sont des permutations du groupe de symétrie de $\Sigma$, on en déduit que $\rho_{(p, r)}$ en est une aussi.


  On a bien que tous les $(\rho''_q)_{q\in Q\times Q'}$ sont des permutations sur les lettres, c'est à dire que l'automate est inversible.
\end{proof}

\begin{prop}[Cloture des inversibles par facteurs]
  Soit $\mathcal{A}$ un automate de Mealy \textbf{inversible} qui se factorise en deux automates $\mathcal{A}_1$ et $\mathcal{A}_2$. Alors ces deux automates sont aussi inversibles.
\end{prop}

\begin{proof}
  Soient $\mathcal{A}=\left(\mathcal{Q\cdot Q'}, \Sigma, \delta, \rho\right)$, $\mathcal{A}_1=\left(\mathcal{Q}, \Sigma, \delta', \rho'\right)$ et $\mathcal{A}_2=\left(\mathcal{Q'}, \Sigma, \delta'', \rho''\right)$ tel que ci dessus.

  Pour chaque état $(p, r)$ de $\mathcal{A}$, $\rho_{(p, r)}\in S(\Sigma)$. Comme $\mathcal{A}=\mathcal{A}_1\cdot\mathcal{A}_2$, on a l'égalité $\rho_{(p, r)}=\rho'_p\circ\rho''_r$. Puisque $\rho_{(p, r)}$ est une permutation, on en déduit que $\rho'_p$ et $\rho''_r$ en sont aussi, d'où le résultat.
\end{proof}

\end{document}